\documentclass[11pt]{article}

\oddsidemargin=17pt \evensidemargin=17pt
\headheight=9pt     \topmargin=26pt
\textheight=564pt   \textwidth=433.8pt

\usepackage{amsmath, amsthm, amssymb,graphicx,color,enumitem}%,mathrsfs}%,hyperref
\setlist[enumerate]{parsep=0pt plus 4pt,topsep=3pt plus 4pt}
\setlist[itemize]{parsep=0pt plus 4pt,topsep=3pt plus 4pt,itemsep=.5ex}

\leftmargini=5.5ex
\leftmarginii=3.5ex

%for \marginpar to fit optimally
%hoffset=-1.02in
\setlength\marginparwidth{2.2in}
\setlength\marginparsep{1mm}
\newcommand\red[1]{\marginpar{\linespread{.85}\sf%
\vspace{-1.4ex}\footnotesize{\color{red}#1}}}
%newcommand\score[1]{\marginpar{\colorbox{yellow}{#1/3}}\hspace{-1ex}}
\newcommand\score[1]{\marginpar{\vspace{-2ex}\color{blue}{#1/3}}\hspace{-1ex}}
\newcommand\extra[1]{\marginpar{\color{blue}{#1/1}}\hspace{-1ex}}
\newcommand\total[2]{\marginpar{\colorbox{yellow}{\huge #1/#2}}}
\newcommand\collab[1]{\marginpar{\vspace{-11ex}\colorbox{yellow}{#1/3}}\hspace{-1ex}}
\newcommand\magenta[1]{\colorbox{magenta}{\!\!#1\!\!}}
\newcommand\yellow[1]{\colorbox{yellow}{\!\!#1\!\!}}
\newcommand\green[1]{\colorbox{green}{\!\!#1\!\!}}
\newcommand\cyan[1]{\colorbox{cyan}{\!\!#1\!\!}}
\newcommand\rmagenta[2][\vspace{0ex}]{\red{#1\magenta{\phantom{:}}\,: #2}}
\newcommand\ryellow[2][\vspace{0ex}]{\red{#1\yellow{\phantom{:}}\,: #2}}
\newcommand\rgreen[2][\vspace{0ex}]{\red{#1\green{\phantom{:}}\,: #2}}
\newcommand\rcyan[2][\vspace{0ex}]{\red{#1\cyan{\phantom{:}}\,: #2}}

%For separated lists with consecutive numbering
\newcounter{separated}

\newcommand{\excise}[1]{}
\newcommand{\comment}[1]{{$\star$\sf\textbf{#1}$\star$}}
\newcommand{\exnumber}[1]{\noindent\textbf{#1}}

%%%%%%%%%%%%%%%%%%%%%%%%%%%%%%%%%%%%%%%%%%%%%%%%%%%
%                                                 %
% TO ENABLE GRADING, DO NOT ALTER ABOVE THIS LINE %
%                                                 %
%%%%%%%%%%%%%%%%%%%%%%%%%%%%%%%%%%%%%%%%%%%%%%%%%%%

%new math symbols taking no arguments
\newcommand\0{\mathbf{0}}
\newcommand\CC{\mathbb{C}}
\newcommand\FF{\mathbb{F}}
\newcommand\NN{\mathbb{N}}
\newcommand\QQ{\mathbb{Q}}
\newcommand\RR{\mathbb{R}}
\newcommand\ZZ{\mathbb{Z}}
\newcommand\bb{\mathbf{b}}
\newcommand\kk{\Bbbk}
\newcommand\mm{\mathfrak{m}}
\newcommand\xx{\mathbf{x}}
\newcommand\yy{\mathbf{y}}
\newcommand\GL{\mathit{GL}}
\newcommand\pset{\mathcal{P}}
\newcommand\into{\hookrightarrow}
\newcommand\nsub{\trianglelefteq}
\newcommand\onto{\twoheadrightarrow}
\newcommand\minus{\smallsetminus}
\newcommand\goesto{\rightsquigarrow}

% theorem environment
\newtheorem*{theorem}{Theorem}

%redefined math symbols taking no arguments
\newcommand\<{\langle}
\renewcommand\>{\rangle}
\renewcommand\iff{\Leftrightarrow}
\renewcommand\phi{\varphi}
\renewcommand\implies{\Rightarrow}

%new math symbols taking arguments
\newcommand\ol[1]{{\overline{#1}}}
	
%redefined math symbols taking arguments
\renewcommand\mod[1]{\ (\mathrm{mod}\ #1)}

%roman font math operators
\DeclareMathOperator\im{im}

%for easy 2 x 2 matrices
\newcommand\twobytwo[1]{\left[\begin{array}{@{}cc@{}}#1\end{array}\right]}

%for easy column vectors of size 2
\newcommand\tworow[1]{\left[\begin{array}{@{}c@{}}#1\end{array}\right]}
	
	
%%%%%%%%%%%%%%%%%%%%%%%%%%%%%%%%%%%%%%%%%%%%%%%%%%%%%%%%%%%%%%%%%%%%%%
\begin{document}%%%%%%%%%%%%%%%%%%%%%%%%%%%%%%%%%%%%%%%%%%%%%%%%%%%%%%
%%%%%%%%%%%%%%%%%%%%%%%%%%%%%%%%%%%%%%%%%%%%%%%%%%%%%%%%%%%%%%%%%%%%%%


\title{\mbox{}\\[-8ex]Functions Solutions\normalsize
	\\[-2.5ex]}
\author{Solutions by: Raindrops\_drop \\[1ex]
	\\[-1ex]}
\date{February 13, 2026} 
\maketitle

\vspace{-3ex}%
\noindent

\exnumber{Exercises}

\exnumber{1.} Let $f: A \to B$. Let $A_0 \subset A$ and $B_0 \subset B$.
\begin{enumerate}[label=(\alph*)]
	\item Show that $A_0 \subset f^{-1}(f(A_0))$ and that equality holds if $f$ is 
	injective.
	\begin{proof}
		Note that if $A_0$ is empty, then $A_0 \subseteq f^{-1}(f(A_0))$ trivially.
		Thus we assume that $A_0$ is non-empty. Let $x \in A_0$. Then, $f(x) \in f(A_0)$
		by definition. Since $f(x) \in f(A_0)$, we have that $x \in f^{-1}(f(A_0))$.
		Therefore, $A_0 \subseteq f^{-1}(f(A_0))$.
		
		Let $x \in f^{-1}(f(A_0))$. By definition, $f(x) \in f(A_0)$. In particular, this
		implies that $f(x) = b$ such that $b \in f(A_0)$. Since it may be that $f$ is 
		not injective, we cannot conclude that $x \in A_0$. This is because there
		may exist some $y \in A_0$ such that $f(y) = f(x) \in f(A_0)$ and $y \neq x$.
		
		If, however, $f$ is injective, then it must be that $y = x$ since $f(y) = f(x)$.
		In particular, this would imply that $x \in A_0$ must be true since $f(x) \in f(A_0)$.
		Therefore, we've shown that equality holds if $f$ is injective.
	\end{proof}
	 
	\item Show that $f(f^{-1}(B_0)) \subset B_0$ and that equality holds if $f$ is 
	surjective. 
	\begin{proof}
		If $f(f^{-1}(B_0))$ is empty, then it is trivially a subset of $B_0$. Thus we 
		will assume that $f(f^{-1}B_0)$ is non-empty. Let $x \in f(f^{-1} B_0)$.
		By definition, $x = f(a)$ for at least one $a \in f^{-1}(B_0)$. Since 
		$a \in f^{-1}(B_0)$, it must be that $f(a) \in B_0$. Therefore, $x \in B_0$
		and we can conclude that $f(f^{-1} B_0) \subseteq B_0$.
		
		Let $x \in B_0$. Since $f$ is not necessarily surjective, it might be that 
		there exists no $a \in A$ such that $f(a) = x$. As such, $f^{-1}(x) = \emptyset$,
		and therefore, $f^{-1}(x) \notin f^{-1}(B_0)$. Furthermore, since 
		$f^{-1}(x) \notin f^{-1}(B_0)$, it is also true that $x \notin f(f^{-1}(B_0))$.
		
		Thus, assume that $f$ is surjective. Then, there exists some $a \in A$ such that
		$f(a) = x$. In particular, since $x \in B_0$, $a \in f^{-1}(B_0)$. Furthermore,
		we conclude that $f(f^{-1}(B_0))$ contains $x$. Therefore, 
		$B_0 \subseteq f(f^{-1}(B_0))$ and thus equality holds if $f$ is surjective.
	\end{proof}
\end{enumerate}

\pagebreak
\exnumber{2.} Let $f: A \to B$ and let $A_i \subset A$ and $B_i \subset B$ for $i = 0$
and $i = 1$. Show that $f^{-1}$ preserves inclusions, unions, intersections, and differences
of sets:
\begin{enumerate}[label=(\alph*)]
	\item $B_0 \subset B_1 \implies f^{-1}(B_0) \subset f^{-1}(B_1)$.
	\begin{proof}
		Let $B_0 \subseteq B_1$. If $B_0$ is the empty set, then note that $f^{-1}(B_0)$
		is also the empty set, since $f$ is a rule of assignment, and as such every element
		of $A$ must be mapped to at least one element in $B$. Therefore, it is trivial to note
		that $f^{-1}(B_0) \subseteq f^{-1}(B_1)$. We arrive at the same conclusion if 
		$B_1$ is the empty set. Thus, we will assume that $B_0$ and $B_1$ are non-empty.
		
		If $f^{-1}(B_0)$ is empty, then it is trivial to note that $f^{-1}(B_0) \subseteq 
		f^{-1}(B_1)$. Similarly if $f^{-1}(B_1)$ is empty. Thus assume that 
		$f^{-1}(B_0)$ is non-empty. Let $x \in f^{-1}(B_0)$ and note that this implies 
		that $f(x) \in B_0$ by definition. Since $B_0 \subseteq B_1$, we have that 
		$f(x) \in B_1$. In particular, this implies that $x \in f^{-1}(B_1)$ by definition.
		Therefore, we conclude that $f^{-1}(B_0) \subseteq f^{-1}(B_1)$.
	\end{proof}
	\item $f^{-1}(B_0 \cup B_1) = f^{-1}(B_0) \cup f^{-1}(B_1)$.
	\begin{proof}
		If $f^{-1}(B_0 \cup B_1)$ is empty, this implies that there exists no 
		$a \in A$ such that $f(a) \in B_0 \cup B_1$. In particular, for any $a \in A$,
		$f(a) \notin B_0$ and $f(a) \notin B_1$. Thus, for any $a \in A$, 
		$a \notin f^{-1}(B_0)$ and $a \notin f^{-1}(B_1)$. Therefore, $f^{-1}(B_0)
		\cup f^{-1}(B_1)$ is empty, and thus the equality in our claim holds.
		
		If $f^{-1}(B_0) \cup f^{-1}(B_1)$ is empty, then there exists no $a \in A$
		such that $f(a) \in B_0$ or $f(a) \in B_1$ or both. This implies that for
		all $a \in A$, $f(a) \notin B_0$ and $f(a) \notin B_1$. In particular,
		$f(a) \notin B_0 \cup B_1$. As such, for all $a \in A$, $a \notin f^{-1}(B_0 \cup B_1)$.
		Therefore, $f^{-1}(B_0 \cup B_1)$ is empty and the equality in our claim is 
		true.
		
		Since we've shown that our claim holds when either of the sets is empty, we will
		assume that both sets are non-empty.
		
		Let $a \in f^{-1}(B_0 \cup B_1)$. By definition, $f(a) \in B_0 \cup B_1$.
		Therefore, either $f(a) \in B_0$, $f(a) \in B_1$, or both. If $f(a) \in B_0$,
		note that it implies that $a \in f^{-1}(B_0)$. In particular, we have that 
		$a \in f^{-1}(B_0) \cup f^{-1}(B_1)$. Thus, we conclude that 
		$f^{-1}(B_0 \cup B_1) \subseteq f^{-1}(B_0) \cup f^{-1}(B_1)$. We arrive at 
		the same conclusion by using an analogous argument for the other cases.
		
		Let $a \in f^{-1}(B_0) \cup f^{-1}(B_1)$. Note that either $a \in f^{-1}(B_0)$,
		$a \in f^{-1}(B_1)$, or both. If $a \in f^{-1}(B_0)$, then $f(a) \in B_0$.
		Note that $B_0 \subseteq B_0 \cup B_1$. Therefore, $f(a) \in B_0 \cup B_1$.
		In particular, $a \in f^{-1}(B_0 \cup B_1)$ and we can conclude that 
		$f^{-1}(B_0) \cup f^{-1}(B_1) \subseteq f^{-1}(B_0 \cup B_1)$. We will arrive
		at this same conclusion by using an analogous argument for the other cases.
		
		Therefore, we conclude that $f^{-1}(B_0 \cup B_1) = f^{-1}(B_0) \cup f^{-1}(B_1)$.
	\end{proof}
	\pagebreak
	\item $f^{-1}(B_0 \cap B_1) = f^{-1}(B_0) \cap f^{-1}(B_1)$.
	\begin{proof}
		If $f^{-1}(B_0 \cap B_1)$ is empty, then it implies that there exists no $a \in A$
		such that $f(a) \in B_0 \cap B_1$. Since $f(a) \notin B_0 \cap B_1$, it must
		be that either $f(a) \notin B_0$, $f(a) \notin B_1$, or both. If $f(a) \notin B_0$,
		then it must be that $a \notin f^{-1}(B_0)$. In particular, since $
		a \notin f^{-1}(B_0)$, $a \notin f^{-1}(B_0) \cap f^{-1}(B_1)$. Since this is 
		an argument for all $a \in A$, we conclude that $f^{-1}(B_0) \cap f^{-1}(B_1)$
		must be empty. Thus, the equality in our claim holds. We arrive at this same 
		conclusion for the other cases by using an analogous argument.
		
		If $f^{-1}(B_0) \cap f^{-1}(B_1)$ is empty, then it implies that there exists
		no $a \in A$ such that $a \in f^{-1}(B_0) \cap f^{-1}(B_1)$. Since for all
		$a \in A$, $a \notin f^{-1}(B_0) \cap f^{-1}(B_1)$, it implies that 
		either $a \notin f^{-1}(B_0)$, $a \notin f^{-1}(B_1)$, or both. 
		If $a \notin f^{-1}(B_0)$, then $f(a) \notin B_0$. In particular, it implies 
		that $f(a) \notin B_0 \cap B_1$. As such, $a \notin f^{-1}(B_0 \cap B_1)$.
		Since this is true for all $a \in A$, we conclude that $f^{-1}(B_0 \cap B_1)$
		is empty. Thus, the equality in our claim holds. We can arrive at this same
		conclusion by following an analogous argument for the other cases.
		
		Since our claim holds if either of the sets is empty, we will assume that both
		of them are non-empty.
		
		Let $a \in f^{-1}(B_0 \cap B_1)$. Therefore, $f(a) \in B_0 \cap B_1$.
		This implies that $f(a) \in B_0$ and $f(a) \in B_1$. In particular, it must be
		that $a \in f^{-1}(B_0)$ and $a \in f^{-1}(B_1)$. Therefore, 
		$a \in f^{-1}(B_0) \cap f^{-1}(B_1)$, and we conclude that 
		$f^{-1}(B_0 \cap B_1) \subseteq f^{-1}(B_0) \cap f^{-1}(B_1)$.
		
		Let $a \in f^{-1}(B_0) \cap f^{-1}(B_1)$. Then, it must be that $a \in f^{-1}(B_0)$
		and $a \in f^{-1}(B_1)$. In particular, we have that $f(a) \in B_0$ and 
		$f(a) \in B_1$. Therefore, $f(a) \in B_0 \cap B_1$ which implies that 
		$a \in f^{-1}(B_0 \cap B_1)$. Hence, we conclude that
		$f^{-1}(B_0 \cap B_1) = f^{-1}(B_0) \cap f^{-1}(B_1)$. 
	\end{proof}
	\item $f^{-1}(B_0 - B_1) = f^{-1}(B_0) - f^{-1}(B_1)$.
	\begin{proof}
		If $f^{-1}(B_0 - B_1)$ is empty, then there exists no $a \in A$ such that 
		$a$ is an element of $f^{-1}(B_0 - B_1)$. This implies that $f(a) \notin B_0 - B_1$
		for all $a \in A$. Since $f(a) \notin B_0 - B_1$, it must be that either
		$f(a) \notin B_0$ or $f(a) \in B_0 \cap B_1$. Note that if $f(a) \notin B_0$,
		then $a \notin f^{-1}(B_0)$ and thus $a \notin f^{-1}(B_0) - f^{-1}(B_1)$
		for all $a \in A$. Thus, $f^{-1}(B_0) - f^{-1}(B_1)$ must be empty, and therefore
		the equality in our claim holds. If $f(a) \in B_0 \cap B_1$, note 
		that $a \in f^{-1}(B_0 \cap B_1)$. In particular, this implies that 
		$a \notin f^{-1}(B_0) - f^{-1}(B_1)$ by definition. Hence, $f^{-1}(B_0) -
		f^{-1}(B_1)$ is empty and our claim holds.
		
		If $f^{-1}(B_0) - f^{-1}(B_1)$ is empty, then there exists no $a \in A$ such that
		$a$ is an element of $f^{-1}(B_0) - f^{-1}(B_1)$. In particular, since 
		$a \notin f^{-1}(B_0) - f^{-1}(B_1)$, it must be that either 
		$a \notin f^{-1}(B_0)$ or $a \in f^{-1}(B_0) \cap f^{-1}(B_1)$. If $a \notin
		f^{-1}(B_0)$, then $f(a) \notin B_0$. In particular, $f(a) \notin B_0 - B_1$.
		Hence, $a \notin f^{-1}(B_0 - B_1)$. Since this is true for all $a \in A$,
		it must be that $f^{-1}(B_0 - B_1)$ is empty and thus our claim holds.
		If $a \in f^{-1}(B_0) \cap f^{-1}(B_1)$, then note that $a \notin f^{-1}(B_0 \cap B_1)$
		by problem (c). As such, $f(a) \in B_0 \cap B_1$, and therefore,
		$f(a) \notin B_0 - B_1$. Thus, for all $a \in A$, $a \notin f^{-1}(B_0 - B_1$.
		As such $f^{-1}(B_0 - B_1)$ is empty and our claim holds.
		
		Since our property holds if either of the sets is empty, we will now assume that
		both sets are non-empty.
		
		Let $a \in f^{-1}(B_0 - B_1)$. By definition, $f(a) \in B_0 - B_1$. This implies
		that $f(a) \in B_0$ and $f(a) \notin B_1$. As such, $a \in f^{-1}(B_0)$ and 
		$a \notin f^{-1}(B_1)$. Therefore, $a \in f^{-1}(B_0) - f^{-1}(B_1)$ and we
		conclude that $f^{-1}(B_0 - B_1) \subseteq f^{-1}(B_0) - f^{-1}(B_1)$.
		
		Let $a \in f^{-1}(B_0) - f^{-1}(B_1)$. By definition, $a \in f^{-1}(B_0)$
		and $a \notin f^{-1}(B_1)$. Since $a \in f^{-1}(B_0)$ we have that 
		$f(a) \in B_0$. Similarly, since $a \notin f^{-1}(B_1)$ we have that 
		$f(a) \notin B_1$. Therefore, since $f(a) \in B_0$ and $f(a) \notin B_1$
		we have that $f(a) \in B_0 - B_1$. Hence, $a \in B_0 - B_1$. Therefore,
		$f^{-1}(B_0) - f^{-1}(B_1) \subseteq f^{-1}(B_0 - B_1)$,
		Thus, we conclude that $f^{-1}(B_0 - B_1) = f^{-1}(B_0) - f^{-1}(B_1)$.
	\end{proof}
\end{enumerate}
Show that $f$ preserves inclusions and unions only:
\begin{enumerate}
	\item[(e)] $A_0 \subset A_1 \implies f(A_0) \subset f(A_1)$.
	\begin{proof}
		Let $A_0 \subseteq A_1$. Note that if either $f(A_0)$ or $f(A_1)$ are empty our 
		claim will hold. As such, we will assume that both $A_0$ and $A_1$ are non-empty.
		
		Let $b \in f(A_0)$. This implies that there exists at least one $a \in A_0$
		such that $f(a) = b$. By assumption, $A_0 \subseteq A_1$ and thus it must
		be that $a \in A_1$. Therefore, $f(a) \in A_1$. Since $f(a) = b$ it must
		be that $b \in A_1$. Therefore, $f(A_0) \subseteq f(A_1)$.
	\end{proof}
	\item[(f)] $f(A_0 \cup A_1) = f(A_0) \cup f(A_1)$.
	\begin{proof}
		If $f(A_0 \cup A_1)$ is empty, then it implies that $A_0 \cup A_1$ is empty.
		In particular, this means $A_0$ and $A_1$ are empty and thus 
		$f(A_0)$ and $f(A_1)$ are empty. Hence, our claim holds. 
		If $f(A_0) \cup f(A_1)$ is empty, then it must be that $f(A_0)$ and $f(A_1)$
		are both empty. Hence, $A_0$ and $A_1$ must be empty. Therefore,
		$f(A_0 \cup A_1)$ is empty and our property holds.
		
		Since our property hold if either of the sets are empty, we will assume that 
		both of the sets are non-empty.
		
		Let $x \in f(A_0 \cup A_1)$. By definition, it must be that $x = f(a)$ for 
		some $a \in A_0 \cup A_1$. Thus, it must be that either $a \in A_0$, $a \in A_1$,
		or both. If $a \in A_0$, it implies that $f(a) \in f(A_0)$, i.e., $x \in f(A_0)$.
		In particular, since $x \in f(A_0)$, we have that $x \in f(A_0) \cup f(A_1)$.
		Thus, we conclude that $f(A_0 \cup A_1) \subseteq f(A_0) \cup f(A_1)$.
		
		Let $x \in f(A_0) \cup f(A_1)$. Thus, either $x \in f(A_0)$, $x \in f(A_1)$,
		or both. If $x \in f(A_0)$ we have that there exists at least one $a \in A_0$
		such that $f(a) = x$. In particular, since $a \in A_0$ it is also true that
		$a \in A_0 \cup A_1$. Therefore, $f(a) = x \in f(A_0 \cup A_1)$. Thus,
		we conclude that $f(A_0) \cup f(A_1) \subseteq f(A_0 \cup A_1)$. We arrive
		at the same conclusion by following an analogous argument for the other cases.
		Therefore, we conclude that $f(A_0 \cup A_1) = f(A_0) \cup f(A_1)$.
	\end{proof}
	\pagebreak
	\item[(g)] $f(A_0 \cap A_1) \subset f(A_0) \cap f(A_1)$; show that equality holds
	if $f$ is injective.
	\begin{proof}
		If $f(A_0 \cap A_1)$ is empty, then it is trivial to note that our claim hold
		wether $f(A_0) \cap f(A_1)$ is empty or not.
		
		If $f(A_0) \cap f(A_1)$ is empty, it implies that there is no $a \in A$
		such that $f(a) \in f(A_0)$, or $f(a) \in f(A_1)$, or both. Since $f$ may not be
		injective, it could be that $a \in A$ is an element of either 
		$A_0$, $A_1$, or both. In the case $a$ is an element of both we note that 
		$a \in A_0 \cap A_1$ and therefore $f(a) \in f(A_0 \cap A_1)$. Thus,
		we cannot conclude that $f(A_0) \cap f(A_1) \subset f(A_0 \cap A_1)$
		for general $f$. Thus, let $f$ be injective. Recall either $f(a) \notin f(A_0)$,
		$f(a) \notin f(A_1)$, or both. If $f(a) \notin f(A_0)$ but $f(a) \in f(A_1)$,
		then $a \notin A_0$ and $a \in A_1$ by definition, i.e., $a \notin A_0 \cap A_1$. 
		Since $f$ is injective, there are no other $a' \in A$ distinct from $a$ such that
		$f(a') = f(a)$. Hence, since $a \notin A_0 \cap A_1$, $f(a) \notin f(A_0 \cap A_1)$. 
		Since this is true for any $a \in A$ we conclude that $f(A_0 \cap A_1)$ must 
		be empty and thus our claim holds.
		
		We've shown that our claim holds if either empty is set. Furthermore, we also
		showed that equality holds if $f$ is injective when either set is empty.
		Thus, we need only show our claim holds for the case where both sets are 
		non-empty.
		
		Let $x \in f(A_0 \cap A_1)$. By definition, there exists at least one 
		$a \in A_0 \cap A_1$ such that $f(a) = x$. In particular, for such $a$ we
		have that $a \in A_0$ and $a \in A_1$. We note that $f(a) \in A_0$ and 
		$f(a) \in A_1$ by definition. Since $f(a) = x$ we conclude that 
		$x \in f(A_0) \cap A_1$. Therefore, $f(A_0 \cap A_1) \subseteq f(A_0) \cap f(A_1)$.
		
		Let $x \in f(A_0) \cap f(A_1)$. Thus, $x \in f(A_0)$ and $x \in f(A_1)$.
		As such, there exists at least one $a_0 \in A_0$ such that $f(a_0) = x$
		and at least one $a_1 \in A_1$ such that $f(a_1) = x$. Since $f$ is not injective,
		it may be that $a_0 \neq a_1$. Therefore, it may be that $a_0, a_1 \notin A_0 \cap A_1$.
		This can be true for any $a \in A$ such that $f(a) = x$. Hence, it may be
		that $x \notin f(A_0 \cap A_1)$ for general $f$. If $f$ is injective,
		note that it must be that $a = a_0 = a_1$ as such, since $a_0 \in A_0$ and 
		$a_1 \in A_1$, we have that $a \in A_0 \cap A_1$ (where $a$ stands for any $a \in A$
		that could have mapped onto $x$ through $f$). Therefore, $f(a) \in f(A_0 \cap A_1)$,
		and since $x = f(a)$ we can conclude that $x \in f(A_0 \cap A_1)$. Hence,
		$f(A_0) \cap f(A_1) \subseteq f(A_0 \cap A_1)$, and we can conclude that
		$f(A_0 \cap A_1) = f(A_0) \cap f(A_1)$ for $f$ injective.
	\end{proof}
	\item[(h)] $f(A_0 - A_1) \supset f(A_0) - f(A_1)$; show that equality holds if 
	$f$ is injective.
	\begin{proof}
		Consider the case where $f(A_0 - A_1)$ is empty. Then, it must be that
		$A_0 - A_1$ is empty. Since $A_0 - A_1$ is empty, either $A_0$ is empty
		or $A_0 \subseteq A_1$. If $A_0$ is empty, note that $f(A_0)$ must also be empty.
		As such, $f(A_0) - f(A_1)$ will also be empty, showing that our claim holds.
		If $A_0 \subseteq A_1$, by problem (e) we have that $f(A_0) \subseteq f(A_1)$.
		Therefore, $f(A_0) - f(A_1)$ must be empty. Hence, our claim holds. 
		
		Consider now the case where $f(A_0) - f(A_1)$ is empty. This implies that
		either $f(A_0)$ is empty or $f(A_0) \subseteq f(A_1)$ where $f(A_0)$ is non-empty. 
		If $f(A_0)$ is empty, note that it implies that $A_0$ is empty. Hence, $A_0 - A_1$
		must also be empty. Therefore, $f(A_0 - A_1)$ is empty and our property holds. 
		Thus, assume that $f(A_0) \subseteq f(A_1)$ where $f(A_0)$ is non-empty. Consider
		$x \in f(A_0)$, and note that there exists at least one $a_0 \in A_0$ such
		that $f(a_0) = x$. Furthermore, since $f(A_0) \subseteq f(A_1)$ we note that
		$x \in f(A_1)$. In particular, there exists at least one $a_1 \in A_1$ such
		that $f(a_1) = x$. Since $f$ may not be injective, it could be that $a_0 \neq a_1$
		and as such it may be that $a_0 \in A_0 - A_1$. Therefore, we may have that
		$x = f(a_0) \in f(A_0 - A_1)$ implying this set is non-empty. Thus, our claim
		still holds for general $f$. Note, however, if $f$ were injective then
		$a_0 = a_1$ must be true since $f(a_0) = x = f(a_1)$. Therefore, it must be 
		that $a_0 \notin A_0 - A_1$. Therefore, it must be that $A_0 - A_1$ is empty, 
		and therefore $f(A_0 - A_1)$ is also empty, showing that equality holds if $f$
		is injective.
		
		We've shown that our claim holds when either of the sets is empty. Thus, we 
		will now assume that both sets are non-empty.
		
		Let $x \in f(A_0) - f(A_1)$. This implies that $x \in f(A_0)$ and $x \notin f(A_1)$.
		Therefore, there exists at least one $a_0 \in A_0$ such that $f(a_0) = x$
		and that there exists no $a_1 \in A_1$ such that $f(a_1) = x$. In particular,
		this implies that $a_0 \neq a_1$ since the element they are mapped onto by $f$
		is different. In particular, $a_0 \in A_0$ and $a_0 \notin A_1$. Therefore,
		$a_0 \in A_0 - A_1$. By definition, $x = f(a_0) \in f(A_0 - A_1)$. Thus,
		we conclude that $f(A_0) - f(A_1) \subseteq f(A_0 - A_1)$.
		
		Let $x \in f(A_0 - A_1)$. This implies that there exists at least one 
		$a \in A_0 - A_1$ such that $f(a) = x$. In particular, we have that $a \in A_0$
		and $a \notin A_1$. Note that since $f$ is not injective, it could be that
		there exists some $a' \in A_1$ such that $a' \notin A_0$ and $f(a') = x$.
		As such, it would imply that $x \in f(A_0)$ and $x \in f(A_1)$. Therefore,
		it would not be possible for $x$ to be an element of $f(A_0) - f(A_1)$.
		If $f$ were injective, then it must be that $a = a'$ since they both map onto $x$.
		In particular, it must be that $a' \in A_0$ and $a' \in A_1$ since that 
		is how we defined $a$. As such, $x \in f(A_0)$ and $x \notin f(A_1)$,
		and we conclude that $x \in f(A_0) - f(A_1)$. Hence, if $f$ is injective
		we also have that $f(A_0 - A_1) \subseteq f(A_0) - f(A_1)$ and thus
		$f(A_0 - A_1) = f(A_0) - f(A_1)$.
	\end{proof}
\end{enumerate}

\exnumber{3.} Show that (b), (c), (f), and (g) of Exercise 2 hold for arbitrary unions 
and intersections.
\begin{enumerate}
	\item[(b)] $f^{-1} \big(\bigcup_{B \in \mathcal{B}} B\big) = 
	\bigcup_{B \in \mathcal{B}} f^{-1}(B)$.
	\begin{proof}
		Let $f^{-1}\big(\bigcup_{B \in \mathcal{B}} B\big)$ be empty. By definition,
		we have that there exists no $a \in A$ such that 
		$f(a) \in \big(\bigcup_{B \in \mathcal{B}} B\big)$. As such, it must be 
		that $f(a)$ is not an element of $B$ for all $B \in \mathcal{B}$. In particular,
		this implies that $a \notin f^{-1}(B)$ for all $B \in \mathcal{B}$. Thus, note
		that this allows us to conclude that $a \notin \bigcup_{B \in \mathcal{B}} f^{-1}(B)$.
		Since this is true for all $a \in A$, we conclude that 
		$\bigcup_{B \in \mathcal{B}} f^{-1}(B)$ is empty and therefore our claim holds.
		
		Let $\bigcup_{B \in \mathcal{B}} f^{-1}(B)$ be empty. This implies that 
		$f^{-1}(B)$ is empty for all $B \in \mathcal{B}$. By definition, this means 
		that there exists no $a \in A$ such that $f(a) \in B$ for all $B \in \mathcal{B}$.
		In particular, we note that $f(a) \notin \bigcup_{B \in \mathcal{B}} B$.
		Therefore, it must be that $a \notin f^{-1}\big(\bigcup_{B \in \mathcal{B}} B\big)$.
		Since this is true for all $a \in A$, we conclude that 
		$f^{-1}\big(\bigcup_{B \in \mathcal{B}} B\big)$ is empty, and thus our claim holds.
		
		Thus, we've shown that if either of the sets is empty our property will hold. We
		need only check for the case where both sets are non-empty.
		
		Let $a \in f^{-1}\big(\bigcup_{B \in \mathcal{B}} B\big)$. By definition,
		$f(a) \in \bigcup_{B \in \mathcal{B}} B$. This implies that 
		${f(a) \in B}$ for at least one $B \in \mathcal{B}$. In particular, note that
		$a \in f^{-1}(B)$. Thus, it must be that ${f(a) \in \bigcup_{B \in \mathcal{B}} B}$.
		Hence, we conclude that $a \in \bigcup_{B \in \mathcal{B}} f^{-1}(B)$.
		Therefore, $f^{-1}\big(\bigcup_{B \in \mathcal{B}} B\big) \subseteq 
		\bigcup_{B \in \mathcal{B}} f^{-1}(B)$.
		
		Let $a \in \bigcup_{B \in \mathcal{B}} f^{-1}(B)$. This implies that 
		${a \in f^{-1}(B)}$ for at least one $B \in \mathcal{B}$. Therefore,
		$f(a) \in B$ for at least one $B \in \mathcal{B}$. In particular, 
		$f(a) \in \bigcup_{B \in \mathcal{B}} B$. Thus, we can conclude that
		$a \in f^{-1}\big(\bigcup_{B \in \mathcal{B}} B\big)$. Hence,
		$\bigcup_{B \in \mathcal{B}} f^{-1}(B) = f^{-1}\big(\bigcup_{B \in \mathcal{B}} B\big)$.
	\end{proof}
	\item[(c)] $f^{-1} \big(\bigcap_{B \in \mathcal{B}} B\big) = 
	\bigcap_{B \in \mathcal{B}} f^{-1}(B)$.
	\begin{proof}
		Let $f^{-1}\big(\bigcap_{B \in \mathcal{B}} B\big)$ be empty. This implies
		that there exists no $a \in A$ such that $f(a) \in \bigcap_{B \in \mathcal{B}} B$.
		By definition, it must be that $f(a) \notin B$ for at least one ${B \in \mathcal{B}}$.
		Thus, $a \notin f^{-1}(B)$ for at least one $B \in \mathcal{B}$. In particular, 
		it must be that $a \notin \bigcap_{B \in \mathcal{B}} f^{-1}(B)$. Since this is 
		true for all $a \in A$, we conclude that $\bigcap_{B \in \mathcal{B}} f^{-1}(B)$
		must be empty, thus our claim holds.
		
		Let $\bigcap_{B \in \mathcal{B}} f^{-1}(B)$ be empty. This implies that
		$f^{-1}(B)$ is empty for at least one $B \in \mathcal{B}$ or that 
		$f^{-1}(B) \cap f^{-1}(B') = \emptyset$ for all $B, B' \in \mathcal{B}$.
		If at least one $f^{-1}(B)$ is empty, then it implies that there exists no
		$a \in A$ such that $f(a) \in B$. In particular, since $f(a) \notin B$,
		it must be that $f(a) \notin \bigcap_{B \in \mathcal{B}} B$. Therefore,
		$a \notin f^{-1}\big(\bigcap_{B \in \mathcal{B}} B\big)$ and thus this set must
		be empty which allows our claim to hold. If $f^{-1}(B) \cap f^{-1}(B') = \emptyset$ 
		for all $B, B' \in \mathcal{B}$, note if $a \in f^{-1}(B)$, then it must be
		that $a \notin f^{-1}(B')$. In particular, this implies that 
		$f(a) \in B$ and $f(a) \notin B'$, as such $f(a) \notin B \cap B'$.
		Since $B, B' \in \mathcal{B}$, we conclude that 
		$f(a) \notin \bigcap_{B \in \mathcal{B}} B$. Therefore,
		$a \notin f^{-1}\big(\bigcap_{B \in \mathcal{B}} B\big)$. Since this is true
		for all $a \in A$, we conclude that $f^{-1}\big(\bigcap_{B \in \mathcal{B}} B\big)$
		must be empty, thus our claim holds.
		
		Since we have shown our property holds when either of the sets is empty, we need
		only show that our property holds when both sets are non-empty.
		
		Let $a \in f^{-1}\big(\bigcap_{B \in \mathcal{B}} B\big)$. By definition,
		$f(a) \in \bigcap_{B \in \mathcal{B}} B$. Thus, we have that 
		$f(a) \in B$ for all $B \in \mathcal{B}$. In particular, it must be that 
		$a \in f^{-1}(B)$ for all $B \in \mathcal{B}$. Hence, 
		$a \in \bigcap_{B \in \mathcal{B}} f^{-1}(B)$ by definition. Therefore we 
		conclude that $f^{-1}\big(\bigcap_{B \in \mathcal{B}} B\big) \subseteq
		\bigcap_{B \in \mathcal{B}} f^{-1}(B)$.
		
		Let $a \in \bigcap_{B \in \mathcal{B}} f^{-1}(B)$. By definition,
		$a \in f^{-1}(B)$ for all $B \in \mathcal{B}$. This implies that 
		$f(a) \in B$ for all $B \in \mathcal{B}$. By definition, it must be 
		that $f(a) \in \bigcap_{B \in \mathcal{B}} B$. Therefore, we conclude 
		that $a \in f^{-1} \big(\bigcap_{B \in \mathcal{B}} B\big)$.
		Hence we have that $f^{-1} \big(\bigcap_{B \in \mathcal{B}} B\big) = 
		\bigcap_{B \in \mathcal{B}} f^{-1}(B)$.
	\end{proof}
	\item[(f)] $f\big(\bigcup_{A \in \mathcal{A}} A\big) = 
	\bigcup_{A \in \mathcal{A}} f(A)$.
	\begin{proof}
		Let $f\big(\bigcup_{A \in \mathcal{A}} A\big)$ be empty. This can only be true
		if $\bigcup_{A \in \mathcal{A}} A$ is empty itself. Thus it must be that 
		$A = \emptyset$ for all $A \in \mathcal{A}$. As such, $f(A)$ is empty 
		for all $A \in \mathcal{A}$. Therefore, $\bigcup_{A \in \mathcal{A}} f(A)$
		is empty and our property holds.
		
		Let $\bigcup_{A \in \mathcal{A}} f(A)$ be empty. This implies that 
		$f(A)$ is empty for all $A \in \mathcal{A}$. Note that $f(A)$ can only 
		be empty if $A$ itself is empty. Therefore, $A = \emptyset$ for all 
		$A \in \mathcal{A}$, and as such $\bigcup_{A \in \mathcal{A}} A$ must also
		be empty. Thus, $f\big(\bigcup_{A \in \mathcal{A}} A\big)$ is empty and our 
		property holds.
		
		We have shown that our property holds when either of the sets is empty. Thus,
		we need only show that the claim holds when both sets are non-empty.
		
		Let $x \in f\big(\bigcup_{A \in \mathcal{A}} A\big)$. Thus, there exists
		at least one $a \in \bigcup_{A \in \mathcal{A}} A$ such that ${f(a) = x}$.
		By definition, $a \in A$ for at least one $A \in \mathcal{A}$. As such,
		$f(a) \in f(A)$ for at least one $A \in \mathcal{A}$. In particular, this 
		implies that $f(a) \in \bigcup_{A \in \mathcal{A}} f(A)$. Since 
		$f(a) = x$ we note that $x \in \bigcup_{A \in \mathcal{A}} f(A)$.
		Therefore, $f\big(\bigcup_{A \in \mathcal{A}} A\big) \subseteq 
		\bigcup_{A \in \mathcal{A}} f(A)$.
		
		Let $x \in \bigcup_{A \in \mathcal{A}} f(A)$. By definition, 
		$x \in f(A)$ for at least one $A \in \mathcal{A}$, let $A$ be such set.
		Thus, there exists at least one $a \in A$ such that $f(a) = x$.
		Note that since $a \in A$ and $A \in \mathcal{A}$, it must be that 
		$a \in \bigcup_{A \in \mathcal{A}} A$. Therefore, 
		$x = f(a) \in f\big(\bigcup_{A \in \mathcal{A}} A\big)$.
		Thus, we conclude that $f\big(\bigcup_{A \in \mathcal{A}} A\big) = 
		\bigcup_{A \in \mathcal{A}} f(A)$.
	\end{proof}
	\item[(g)] $f\big(\bigcap_{A \in \mathcal{A}} A\big) \subseteq 
	\bigcap_{A \in \mathcal{A}} f(A)$; show that equality holds if $f$ is injective.
	\begin{proof}
		Let $f\big(\bigcap_{A \in \mathcal{A}} A\big)$ be empty. This can only be
		true if $\bigcap_{A \in \mathcal{A}} A$ is itself empty. Thus, it must be 
		that at least one $A \in \mathcal{A}$ is empty, or $A \cap A' = \emptyset$
		for all $A, A' \in \mathcal{A}$. If at least one $A \in \mathcal{A}$ is empty,
		note that $f(A)$ is empty for such $A$. Note that this implies that 
		$\bigcap_{A \in \mathcal{A}} f(A)$ is empty and thus our property holds.
		If $A \cap A' = \emptyset$ for all $A, A' \in \mathcal{A}$, note that we 
		cannot say anything about $f(A) \cap f(A')$. This is because $f$ is not 
		necessarily injective, as such there may exist $a \in A$ and $a' \in A'$
		such that $f(a) = f(a')$. Thus, it could be that $f(A) \cap f(A')$ is non-empty.
		Thus, at most we can only conclude that $f\big(\bigcap_{A \in \mathcal{A}} A\big)
		\subseteq \bigcap_{A \in \mathcal{A}} f(A)$. If $f$ was injective,
		note that such $a$ and $a'$ cannot exists since $f(a) = f(a')$ only if 
		$a = a'$. Thus, we note that $f(A) \cap f(A')$ must be empty for 
		all $A, A' \in \mathcal{A}$. Therefore, $\bigcap_{A \in \mathcal{A}} f(A)$
		must be empty and equality holds.
		
		Let $\bigcap_{A \in \mathcal{A}} f(A)$ be empty. This implies that 
		either $f(A)$ is empty for at least one $A \in \mathcal{A}$ or 
		$f(A) \cap f(A') = \emptyset$ for all $A, A' \in \mathcal{A}$.
		If $f(A)$ is empty for at least one $A \in \mathcal{A}$ note that 
		$A$ must be empty. Thus, $\bigcap_{A \in \mathcal{A}} A$ must also be empty,
		and therefore $f\big(\bigcap_{A \in \mathcal{A}} A\big)$ is empty and 
		our property holds. If $f(A) \cap f(A') = \emptyset$ for all $A, A' \in \mathcal{A}$,
		then for any $a$ such that $f(a) \in f(A)$ it must be that $f(a) \notin f(A')$.
		Thus, it must be that $a \in A$ and $a \notin A'$ by definition. Since 
		$A, A' \in \mathcal{A}$ we note that $a \notin \bigcap_{A \in \mathcal{A}} A$.
		Therefore, $f(a) \notin f\big(\bigcap_{A \in \mathcal{A}} A\big)$. Since this 
		is true for all $a$ we conclude that $f\big(\bigcap_{A \in \mathcal{A}} A\big)$
		must be empty and our property holds.
		
		Since we have shown that our claim holds when either of the sets is empty,
		and furthermore that equality holds when $f$ is injective, we need only show
		that our claim holds when both sets are non-empty.
		
		Let $x \in f\big(\bigcap_{A \in \mathcal{A}} A\big)$. By definition,
		there exists at least one $a \in \bigcap_{A \in \mathcal{A}} A$ such that
		$f(a) = x$. Since $a \in \bigcap_{A \in \mathcal{A}} A$, it must be that 
		$a \in A$ for all $A \in \mathcal{A}$. In particular, $f(a) \in f(A)$
		for all $A \in \mathcal{A}$. By definition, 
		$x = f(a) \in \bigcap_{A \in \mathcal{A}} f(A)$. Thus, we conclude that 
		$f\big(\bigcap_{A \in \mathcal{A}} A\big) \subseteq \bigcap_{A \in \mathcal{A}} f(A)$.
		
		Let $x \in \bigcap_{A \in \mathcal{A}} f(A)$. By definition, 
		$x \in f(A)$ for all $A \in \mathcal{A}$. Consider $A, A' \in \mathcal{A}$.
		Note that there exists at least one $a \in A$ and one $a' \in A'$ such that 
		$f(a) = x = f(a')$. Since it may not be true that $f$ is injective, we may 
		have that $a \neq a'$. In particular, note that $a, a' \notin A \cap A'$.
		Thus, we cannot conclude that there exists some $a'' \in \bigcap_{A \in \mathcal{A}} A$
		such that $f(a'') = x$ since it could be that all $a \in A$ for all $A \in \mathcal{A}$
		such that $f(a) = x$ are distinct. If, however, $f$ is injective, note it must 
		be that $a = a'$ since $f(a) = x = f(a')$. Since this is true for all
		$A, A' \in \mathcal{A}$ we conclude that $a \in \bigcap_{A \in \mathcal{A}} A$.
		Thus, $x = f(a) = f\big(\bigcap_{A \in \mathcal{A}} A\big)$.
		Therefore, $f\big(\bigcap_{A \in \mathcal{A}} A\big) \subseteq 
		\bigcap_{A \in \mathcal{A}} f(A)$ if $f$ is injective.
		
	\end{proof}
\end{enumerate}

\exnumber{4.} Let $f: A \to B$ and $g: B \to C$.
\begin{enumerate}[label=(\alph*)]
	\item If $C_0 \subset C$, show that $(g \circ f)^{-1}(C_0) = f^{-1}(g^{-1}(C_0))$.
	\begin{proof}
		Let $C_0 \subseteq C$. If $(g \circ f)^{-1}(C_0)$ is empty, then by definition
		there exists no $a \in A$ such that $(g \circ f)(a) \in C_0$. In particular,
		we have that $g(f(a)) \notin C_0$. Furthermore, $f(a) \notin g^{-1}(C_0)$ by definition.
		Note $g^{-1}(C_0) \subseteq B$, and thus $a \notin f^{-1}(g^{-1}(C_0)$ by definition.
		Since this is true for all $a \in A$, it must be that $f^{-1}(g^{-1}(C_0))$ is empty
		and our claim holds.
		
		Let $f^{-1}(g^{-1}(C_0))$ be empty.	By definition, there exists no $a \in A$
		such that ${f(a) \in g^{-1}(C_0)}$. Furthermore, this implies that 
		$g(f(a)) = (g \circ f)(a) \notin C_0$. Therefore, it must be that $a \notin (g \circ f)^{-1}(C_0)$.
		Since this is true for all $a \in A$, we conclude that $(g \circ f)^{-1}(C_0)$ is empty
		and our claim holds.
		
		Since we have shown that our claim holds when either of the sets is empty, we need only 
		show that it also holds when both sets are non-empty.
		
		Let $a \in (g \circ f)^{-1}(C_0)$. By definition, $(g \circ f)(a) \in C_0$, and thus
		$g(f(a)) \in C_0$. Note $f(a) \in g^{-1}(C_0)$. Furthermore, $a \in f^{-1}(g^{-1}(C_0))$.
		Therefore, we can conclude ${(g \circ f)^{-1}(C_0) \subseteq f^{-1}(g^{-1}(C_0))}$.
		
		Let $a \in f^{-1}(g^{-1}(C_0))$. By definition, it must be that $f(a) \in g^{-1}(C_0)$.
		Furthermore, note $g(f(a)) \in C_0$. Thus, $(g \circ f)(a) \in C_0$. As such,
		we have that ${a \in (g \circ f)^{-1}(C_0)}$. Therefore, 
		$f^{-1}(g^{-1}(C_0)) \subseteq (g \circ f)^{-1}(C_0)$. Hence, we have shown that 
		our claim holds in the case both sets are non-empty, i.e., 
		${(g \circ f)^{-1}(C_0) = f^{-1}(g^{-1}(C_0))}$.
	\end{proof}
	\item If $f$ and $g$ are injective, show that $g \circ f$ is injective.
	\begin{proof}
		Let $f$ and $g$ be injective. Let $a, a' \in A$ such that 
		$(g \circ f)(a) = (g \circ f)(a') = c$. Thus, we have that 
		$g(f(a)) = g(f(a')) = c$ by definition. Since $g$ is injective
		it must be that $f(a) = f(a')$ since they both map under $g$ onto $c$.
		Furthermore, since $f$ is injective, it must be that $a = a'$.
		Therefore, we conclude that $g \circ f$ is injective, since 
		$(g \circ f)(a) = (g \circ f)(a')$ implies that $a = a'$.
	\end{proof}
	\pagebreak
	\item If $g \circ f$ is injective, what can you say about the injectivity of 
	$f$ and $g$?
	\begin{proof}
		Let $g \circ f$ be injective. Assume for contradiction that $f$ is not injective.
		Consider $g(f(a))$ and $g(f(a'))$ such that $g(f(a)) = g(f(a''))$ where 
		$a, a' \in A$. Since $f$ is not injective, it may be that 
		$f(a) = f(a')$. Thus, it still holds that $g(f(a)) = g(f(a'))$. This is a 
		contradiction since $g \circ f$ is injective. Therefore, we conclude that 
		$f$ must be injective if $g \circ f$ is injective.
		
		Note that $g$ can either be injective or non-injective. If $g$ is injective it 
		still holds that $g \circ f$ is injective by problem (b). Thus, assume $g$ is 
		non-injective. Consider $g(f(a))$ and $g(f(a'))$ such that 
		$g(f(a)) = g(f(a'))$ for $a, a' \in A$. Since $f$ is injective, if $a = a'$ 
		we have that $f(a) = f(a')$ and therefore injectivity of $g \circ f$ still holds.
		If $a \neq a'$, it must be that $f(a) \neq f(a')$. Since $g$ is non-injective
		it is possible for $g(f(a)) = g(f(a'))$. Thus, the injectivity of $g \circ f$
		will still hold. Therefore, since $g \circ f$ can be injective when $g$ is either
		injective or non-injective, we conclude that the injectivity of $g \circ f$
		does not allow us to say anything about the injectivity of $g$.
	\end{proof}
	\item If $f$ and $g$ are surjective, show that $g \circ f$ is surjective.
	\begin{proof}
		Let $f$ and $g$ be surjective. Consider $c \in C$. Since $g$ is surjective, there
		exists some $b \in B$ such that $g(b) = c$. Similarly, since $f$ is surjective
		there exists some $a \in A$ such that $f(a) = b$. In particular, since $f(a) = b$
		and $g(b) = c$, we conclude that $g(f(a)) = c$. Thus, $(g \circ f)(a) = c$.
		Since this is true for any $c \in C$, we conclude that $g \circ f$ must be 
		surjective.
	\end{proof}
	\item If $g \circ f$ is surjective, what can you say about the surjectivity of 
	$f$ and $g$?
	\begin{proof}
		Let $g \circ f$ be surjective. Assume for contradiction that $g$ is not 
		surjective. Then, there exists at least one $c \in C$ such that $g(b) \neq c$
		for all $b \in B$. Note this is a contradiction since $g \circ f$ is surjective.
		Therefore, it must be that $g$ is surjective if $g \circ f$ is surjective.
		
		Note that $f$ need not be surjective. Consider the example where
		$A = \{1, 2, 3\}$, $B = \{1, 2, 3, 4\}$, and $C = \{1, 2, 3\}$.
		Define $f: A \to B$ such that $f(a) = a$, that is $f(1) = 1$, $f(2) = 2$ and 
		$f(3) = 3$. Define $g: B \to C$ such that $g(1) = 1$, $g(2) = 2$, $g(3) = 3$,
		and $g(4) = 3$. Note that $f$ is not surjective, $g$ is surjective,
		and that $g \circ f$ is surjective. Thus, the surjectivity of $g \circ f$
		does not allow us to make any claims about the surjectivity of $f$.
	\end{proof}
	\pagebreak
	\item Summarize your answers to (b)-(c) in the form of a theorem.
	\begin{theorem}
		Let $A, B$, and $C$ be sets. Define $f: A \to B$ and $g: B \to C$. The following
		statements can be said about $f$, $g$, and $g \circ f$:
		\begin{enumerate}
			\item If $f$ and $g$ are surjective, then $g \circ f$ is surjective.
			\item If $g \circ f$ is surjective, then $g$ is surjective.
			\item If $f$ and $g$ are injective, then $g \circ f$ is injective.
			\item If $g \circ f$ is injective, then $f$ is injective.
		\end{enumerate}
	\end{theorem}
\end{enumerate}

\exnumber{5.} In general, let us denote the \textbf{\textit{identity function}} for a 
set $C$ by $i_C$. That is, define $i_C: C \to C$ to be the function given by the rule 
$i_C(x) = x$ for all $x \in C$. Given $f: A \to B$, we say that $h: B \to A$ is a 
\textbf{\textit{right inverse}} for $f$ if $f \circ h = i_B$.
\begin{enumerate}[label=(\alph*)]
	\item Show that if $f$ has a left inverse, $f$ is injective; and if $f$ has a right
	inverse, $f$ is surjective.
	\begin{proof}
		Let $f$ have a left inverse, call it $h: B \to A$. Thus, $h \circ f = i_A$ by 
		definition. Note that $i_A$ is an injective function, since $i_A(x) = i_A(x')$
		it implies that $x = x'$ since $i_A(x) = x$ and $i_A(x')$. By our theorem in
		Exercise 4 part (f), $f$ must be injective.
		
		Let $f$ have a right inverse, call it $h: B \to A$. Thus, $f \circ g = i_B$ by definition.
		Note $i_B$ is surjective. This is because for any $b \in B$ there exists some $b' \in B$
		such that $i_B(b') = b$, namely $b' = b$ by definition. Thus, by Exercise 4 part (f)
		we conclude that $f$ must be surjective.
	\end{proof}
	\item Give an example of a function that has a left inverse but no right inverse.
	
	Define $A = \{1, 2, 3\}$, $B = \{a, b, c, d\}$, and $f: A \to B$ such that 
	$f(1) = a$, $f(2) = b$, and $f(3) = c$. Note $f$ has the left inverse
	$h: B \to A$ defined by $h(a) = 1$, $h(b) = 2$, $h(3) = c$, and $h(d) = 3$.
	Note it is a left inverse since $(h \circ f)(A) = h(f(A)) = A$ and 
	$h \circ f$ is equivalent to $i_A$ by construction. If $f$ had a right inverse,
	then by (a) $f$ should be surjective. Note by construction $f$ is not surjective,
	therefore, $f$ does not have a right inverse.
	 
	\item Give an example of a function that has a right inverse but no left inverse.
	
	Define $A = \{1, 2, 3, 4\}$, $B = \{a, b, c\}$, and $f: A \to B$ by the rule
	$f(1) = a$, $f(2) = b$, $f(3) = c$, and $f(4) = c$. Note $f$ has a right inverse
	$h: B \to A$ defined by $h(a) = 1$, $h(b) = 2$, and $h(c) = 3$. We know $h$ is a 
	right inverse because $(f \circ h)(B) = f(h(B)) = B$ and $(f \circ h)(x) = x$ for 
	all $x \in B$ by construction. If $f$ had a left inverse, then by (a) $f$ should
	be injective. Note by construction $f$ is not surjective, therefore $f$ does not 
	have a left inverse.
	
	\item Can a function have more than one left inverse? More than one right inverse?
	
	Yes, a function can have more than one left inverse and more than one right inverse.
	To see this, consider the examples given in parts (b) and (c). In part (b) note that
	if we had defined $h: B \to A$ such that $h(d) = 1$ instead of $h(d) = 3$ it is 
	still true that $h \circ f = i_A$.
	Similarly, in (c) if we had defined $h: B \to A$ such that $h(c) = 4$ instead of 
	$h(c) = 3$ it is still true that $h \circ f = i_B$. In both of these cases
	we defined a new function $h$ since it had a different rule than our original
	function $h$, and as such both are different inverses for $f$ in each example.
	
	\item Show that if $f$ has both a left inverse $g$ and a right inverse $h$, then
	$f$ is bijective and $g = h = f^{-1}$.
	\begin{proof}
		Let $f$ have both a left inverse $g: B \to A$ and a right inverse $h: B \to A$.
		By problem (a), since $f$ has a left inverse it must be that $f$ is injective.
		Similarly, by problem (a) since $f$ has a right inverse it must be that $f$ is surjective.
		Since $f$ is both injective and surjective we conclude that $f$ is bijective. 
		
		Let $b \in B$ and consider $g(b)$. Since $f$ is bijective, it is also surjective and thus 
		there exists some $a \in A$ such that $f(a) = b$. As such, $g(b) = g(f(a))$. Since $g$ is 
		the left inverse of $f$ note it must be that $g(f(a)) = a$. Similarly, since $b \in B$ we have 
		that $h(b)$ exists. In particular, since $h$ is the right inverse of $f$, it must be that 
		$f(h(b)) = b$. Since $f$ is bijective it is also injective. Therefore, since 
		$f(a) = b$ and $f(h(b)) = b$ it must be that $h(b) = a$. Since $g(b) = a$ and 
		$h(b) = a$ for arbitrary $b \in B$ we note that $g$ and $h$ map the same elements in $B$
		to the same element in $A$ and therefore, since $g$ and $h$ have the same rule for
		mapping and they are defined on the same sets it must be that $g = h$.
		
		In particular, since $f^{-1}: B \to A$ is defined as the function that maps $f^{-1}(b) = a$
		such that $f(a) = b$ where $a$ in the unique element that $f$ maps onto $b$. it must
		be that $f^{-1} = g = h$. This is because, as we showed earlier, both $g$ and $h$ map
		an element $b$ onto the element $a$ such that $f(a) = b$ where $a$ exists and is 
		unique by the fact $f$ is bijective.   
	\end{proof}
\end{enumerate}

\exnumber{6.} Let $f: \RR \to \RR$ be the function $f(x) = x^3 - x$. By restricting the domain
and range of $f$ appropriately, obtain from $f$ a bijective function $g$. Draw the graphs 
of $g$ and $g^{-1}$. (There are several possible choices for $g$.)

We begin by noting that $f$ as defined is not bijective. This is because we have $f$ is not injective
since ${f(1) = f(0) = f(-1) = 0}$. A very simple bijective function $g$ we can create by 
restricting the domain and range of $f$ is $g: \{1\} \to \{0\}$ such that $g(x) = x^3 - x$.
We note that $g^{-1}: \{0\} \to \{1\}$ can be simply defined as $g^{-1}(x) = 1$.
We omit including the graphs of $g$ and $g^{-1}$ since they are just the single element
of $\{1\} \times \{0\}$ and $\{0\} \times \{1\}$ respectively.
 

%%%%%%%%%%%%%%%%%%%%%%%%%%%%%%%%%%%%%%%%%%%%%%%%%%%%%%%%%%%%%%%%%%%%%%
\end{document}%%%%%%%%%%%%%%%%%%%%%%%%%%%%%%%%%%%%%%%%%%%%%%%%%%%%%%%%
%%%%%%%%%%%%%%%%%%%%%%%%%%%%%%%%%%%%%%%%%%%%%%%%%%%%%%%%%%%%%%%%%%%%%%


%%% Various Emacs customizations:
%%% Local Variables:
%%% mode: latex
%%% mode: LaTeX-math
%%% mode: reftex
%%% mode: Tex-PDF
%%% fill-column: 70
%%% indent-tabs-mode: t
%%% TeX-electric-sub-and-superscript: nil
%%% TeX-brace-indent-level: 0
%%% LaTeX-indent-level: 0
%%% LaTeX-item-indent: 0
%%% TeX-master: t
%%% End: