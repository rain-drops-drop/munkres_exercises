\documentclass[11pt]{article}

\oddsidemargin=17pt \evensidemargin=17pt
\headheight=9pt     \topmargin=26pt
\textheight=564pt   \textwidth=433.8pt

\usepackage{amsmath, amsthm, amssymb,graphicx,color,enumitem}%,mathrsfs}%,hyperref
\setlist[enumerate]{parsep=0pt plus 4pt,topsep=3pt plus 4pt}
\setlist[itemize]{parsep=0pt plus 4pt,topsep=3pt plus 4pt,itemsep=.5ex}

\leftmargini=5.5ex
\leftmarginii=3.5ex

%for \marginpar to fit optimally
%hoffset=-1.02in
\setlength\marginparwidth{2.2in}
\setlength\marginparsep{1mm}
\newcommand\red[1]{\marginpar{\linespread{.85}\sf%
\vspace{-1.4ex}\footnotesize{\color{red}#1}}}
%newcommand\score[1]{\marginpar{\colorbox{yellow}{#1/3}}\hspace{-1ex}}
\newcommand\score[1]{\marginpar{\vspace{-2ex}\color{blue}{#1/3}}\hspace{-1ex}}
\newcommand\extra[1]{\marginpar{\color{blue}{#1/1}}\hspace{-1ex}}
\newcommand\total[2]{\marginpar{\colorbox{yellow}{\huge #1/#2}}}
\newcommand\collab[1]{\marginpar{\vspace{-11ex}\colorbox{yellow}{#1/3}}\hspace{-1ex}}
\newcommand\magenta[1]{\colorbox{magenta}{\!\!#1\!\!}}
\newcommand\yellow[1]{\colorbox{yellow}{\!\!#1\!\!}}
\newcommand\green[1]{\colorbox{green}{\!\!#1\!\!}}
\newcommand\cyan[1]{\colorbox{cyan}{\!\!#1\!\!}}
\newcommand\rmagenta[2][\vspace{0ex}]{\red{#1\magenta{\phantom{:}}\,: #2}}
\newcommand\ryellow[2][\vspace{0ex}]{\red{#1\yellow{\phantom{:}}\,: #2}}
\newcommand\rgreen[2][\vspace{0ex}]{\red{#1\green{\phantom{:}}\,: #2}}
\newcommand\rcyan[2][\vspace{0ex}]{\red{#1\cyan{\phantom{:}}\,: #2}}

%For separated lists with consecutive numbering
\newcounter{separated}

\newcommand{\excise}[1]{}
\newcommand{\comment}[1]{{$\star$\sf\textbf{#1}$\star$}}
\newcommand{\exnumber}[1]{\noindent\textbf{#1}}

%%%%%%%%%%%%%%%%%%%%%%%%%%%%%%%%%%%%%%%%%%%%%%%%%%%
%                                                 %
% TO ENABLE GRADING, DO NOT ALTER ABOVE THIS LINE %
%                                                 %
%%%%%%%%%%%%%%%%%%%%%%%%%%%%%%%%%%%%%%%%%%%%%%%%%%%

%new math symbols taking no arguments
\newcommand\0{\mathbf{0}}
\newcommand\CC{\mathbb{C}}
\newcommand\FF{\mathbb{F}}
\newcommand\NN{\mathbb{N}}
\newcommand\QQ{\mathbb{Q}}
\newcommand\RR{\mathbb{R}}
\newcommand\ZZ{\mathbb{Z}}
\newcommand\bb{\mathbf{b}}
\newcommand\kk{\Bbbk}
\newcommand\mm{\mathfrak{m}}
\newcommand\xx{\mathbf{x}}
\newcommand\yy{\mathbf{y}}
\newcommand\GL{\mathit{GL}}
\newcommand\pset{\mathcal{P}}
\newcommand\into{\hookrightarrow}
\newcommand\nsub{\trianglelefteq}
\newcommand\onto{\twoheadrightarrow}
\newcommand\minus{\smallsetminus}
\newcommand\goesto{\rightsquigarrow}

%redefined math symbols taking no arguments
\newcommand\<{\langle}
\renewcommand\>{\rangle}
\renewcommand\iff{\Leftrightarrow}
\renewcommand\phi{\varphi}
\renewcommand\implies{\Rightarrow}

%new math symbols taking arguments
\newcommand\ol[1]{{\overline{#1}}}
	
%redefined math symbols taking arguments
\renewcommand\mod[1]{\ (\mathrm{mod}\ #1)}

%roman font math operators
\DeclareMathOperator\im{im}

%for easy 2 x 2 matrices
\newcommand\twobytwo[1]{\left[\begin{array}{@{}cc@{}}#1\end{array}\right]}

%for easy column vectors of size 2
\newcommand\tworow[1]{\left[\begin{array}{@{}c@{}}#1\end{array}\right]}
	
	
%%%%%%%%%%%%%%%%%%%%%%%%%%%%%%%%%%%%%%%%%%%%%%%%%%%%%%%%%%%%%%%%%%%%%%
\begin{document}%%%%%%%%%%%%%%%%%%%%%%%%%%%%%%%%%%%%%%%%%%%%%%%%%%%%%%
%%%%%%%%%%%%%%%%%%%%%%%%%%%%%%%%%%%%%%%%%%%%%%%%%%%%%%%%%%%%%%%%%%%%%%


\title{\mbox{}\\[-8ex]Functions Solutions\normalsize
	\\[-2.5ex]}
\author{Solutions by: Raindrops\_drop \\[1ex]
	\\[-1ex]}
\date{February 13, 2026} 
\maketitle

\vspace{-3ex}%
\noindent

\exnumber{Exercises}

\exnumber{1.} Let $f: A \to B$. Let $A_0 \subset A$ and $B_0 \subset B$.
\begin{enumerate}[label=(\alph*)]
	\item Show that $A_0 \subset f^{-1}(f(A_0))$ and that equality holds if $f$ is 
	injective.
	\begin{proof}
		Note that if $A_0$ is empty, then $A_0 \subseteq f^{-1}(f(A_0))$ trivially.
		Thus we assume that $A_0$ is non-empty. Let $x \in A_0$. Then, $f(x) \in f(A_0)$
		by definition. Since $f(x) \in f(A_0)$, we have that $x \in f^{-1}(f(A_0))$.
		Therefore, $A_0 \subseteq f^{-1}(f(A_0))$.
		
		Let $x \in f^{-1}(f(A_0))$. By definition, $f(x) \in f(A_0)$. In particular, this
		implies that $f(x) = b$ such that $b \in f(A_0)$. Since it may be that $f$ is 
		not injective, we cannot conclude that $x \in A_0$. This is because there
		may exist some $y \in A_0$ such that $f(y) = f(x) \in f(A_0)$ and $y \neq x$.
		
		If, however, $f$ is injective, then it must be that $y = x$ since $f(y) = f(x)$.
		In particular, this would imply that $x \in A_0$ must be true since $f(x) \in f(A_0)$.
		Therefore, we've shown that equality holds if $f$ is injective.
	\end{proof}
	 
	\item Show that $f(f^{-1}(B_0)) \subset B_0$ and that equality holds if $f$ is 
	surjective. 
	\begin{proof}
		If $f(f^{-1}(B_0))$ is empty, then it is trivially a subset of $B_0$. Thus we 
		will assume that $f(f^{-1}B_0)$ is non-empty. Let $x \in f(f^{-1} B_0)$.
		By definition, $x = f(a)$ for at least one $a \in f^{-1}(B_0)$. Since 
		$a \in f^{-1}(B_0)$, it must be that $f(a) \in B_0$. Therefore, $x \in B_0$
		and we can conclude that $f(f^{-1} B_0) \subseteq B_0$.
		
		Let $x \in B_0$. Since $f$ is not necessarily surjective, it might be that 
		there exists no $a \in A$ such that $f(a) = x$. As such, $f^{-1}(x) = \emptyset$,
		and therefore, $f^{-1}(x) \notin f^{-1}(B_0)$. Furthermore, since 
		$f^{-1}(x) \notin f^{-1}(B_0)$, it is also true that $x \notin f(f^{-1}(B_0))$.
		
		Thus, assume that $f$ is surjective. Then, there exists some $a \in A$ such that
		$f(a) = x$. In particular, since $x \in B_0$, $a \in f^{-1}(B_0)$. Furthermore,
		we conclude that $f(f^{-1}(B_0))$ contains $x$. Therefore, 
		$B_0 \subseteq f(f^{-1}(B_0))$ and thus equality holds if $f$ is surjective.
	\end{proof}
\end{enumerate}

\pagebreak
\exnumber{2.} Let $f: A \to B$ and let $A_i \subset A$ and $B_i \subset B$ for $i = 0$
and $i = 1$. Show that $f^{-1}$ preserves inclusions, unions, intersections, and differences
of sets:
\begin{enumerate}[label=(\alph*)]
	\item $B_0 \subset B_1 \implies f^{-1}(B_0) \subset f^{-1}(B_1)$.
	\begin{proof}
		Let $B_0 \subseteq B_1$. If $B_0$ is the empty set, then note that $f^{-1}(B_0)$
		is also the empty set, since $f$ is a rule of assignment, and as such every element
		of $A$ must be mapped to at least one element in $B$. Therefore, it is trivial to note
		that $f^{-1}(B_0) \subseteq f^{-1}(B_1)$. We arrive at the same conclusion if 
		$B_1$ is the empty set. Thus, we will assume that $B_0$ and $B_1$ are non-empty.
		
		If $f^{-1}(B_0)$ is empty, then it is trivial to note that $f^{-1}(B_0) \subseteq 
		f^{-1}(B_1)$. Similarly if $f^{-1}(B_1)$ is empty. Thus assume that 
		$f^{-1}(B_0)$ is non-empty. Let $x \in f^{-1}(B_0)$ and note that this implies 
		that $f(x) \in B_0$ by definition. Since $B_0 \subseteq B_1$, we have that 
		$f(x) \in B_1$. In particular, this implies that $x \in f^{-1}(B_1)$ by definition.
		Therefore, we conclude that $f^{-1}(B_0) \subseteq f^{-1}(B_1)$.
	\end{proof}
	\item $f^{-1}(B_0 \cup B_1) = f^{-1}(B_0) \cup f^{-1}(B_1)$.
	\begin{proof}
		
	\end{proof}
	\item $f^{-1}(B_0 \cap B_1) = f^{-1}(B_0) \cap f^{-1}(B_1)$.
	\item $f^{-1}(B_0 - B_1) = f^{-1}(B_0) - f^{-1}(B_1)$.
\end{enumerate}
Show that $f$ preserves inclusions and unions only:
\begin{enumerate}
	\item[(e)] $A_0 \subset A_1 \implies f(A_0) \subset f(A_1)$.
	\item[(f)] $f(A_0 \cup A_1) = f(A_0) \cup f(A_1)$.
	\item[(g)] $f(A_0 \cap A_1) \subset f(A_0) \cap f(A_1)$; show that equality holds
	if $f$ is injective.
	\item[(h)] $f(A_0 - A_1) \supset f(A_0) - f(A_1)$; show that equality holds if 
	$f$ is injective.
\end{enumerate}

\exnumber{3.} Show that (b), (c), (f), and (g) of Exercise 2 hold for arbitrary unions 
and intersections.

\exnumber{4.} Let $f: A \to B$ and $g: B \to C$.
\begin{enumerate}[label=(\alph*)]
	\item If $C_0 \subset C$, show that $(g \circ f)^{-1}(C_0) = f^{-1}(g^{-1}(C_0))$.
	\item If $f$ and $g$ are injective, show that $g \circ f$ is injective.
	\item If $g \circ f$ is injective, what can you say about the injectivity of 
	$f$ and $g$?
	\item If $f$ and $g$ are surjective, show that $g \circ f$ is surjective.
	\item If $g \circ f$ is surjective, what can you say about the surjectivity of 
	$f$ and $g$?
	\item Summarize your answers to (b)-(c) in the form of a theorem.
\end{enumerate}

\exnumber{5.} In general, let us denote the \textbf{\textit{identity function}} for a 
set $C$ by $i_C$. That is, define $i_C: C \to C$ to be the function given by the rule 
$i_C(x) = x$ for all $x \in C$. Given $f: A \to B$, we say that $h: B \to A$ is a 
\textbf{\textit{right inverse}} for $f$ if $f \circ h = i_B$.
\begin{enumerate}[label=(\alph*)]
	\item Show that if $f$ has a left inverse, $f$ is injective; and if $f$ has a right
	inverse, $f$ is surjective.
	\item Give an example of a function that has a left inverse but no right inverse.
	\item Give an example of a function that has a right inverse but no left inverse.
	\item Can a function have more than one left inverse? More than one right inverse?
	\item Show that if $f$ has both a left inverse $g$ and a right inverse $h$, then
	$f$ is bijective and $g = h = f^{1}$.
\end{enumerate}

\exnumber{6.} Let $f: \RR \to \RR$ be the function $f(x) = x^3 - x$. By restricting the domain
and range of $f$ appropriately, obtain from $f$ a bijective function $g$. Draw the graphs 
of $g$ and $g^{-1}$. (There are several possible choices for $g$.)
 

%%%%%%%%%%%%%%%%%%%%%%%%%%%%%%%%%%%%%%%%%%%%%%%%%%%%%%%%%%%%%%%%%%%%%%
\end{document}%%%%%%%%%%%%%%%%%%%%%%%%%%%%%%%%%%%%%%%%%%%%%%%%%%%%%%%%
%%%%%%%%%%%%%%%%%%%%%%%%%%%%%%%%%%%%%%%%%%%%%%%%%%%%%%%%%%%%%%%%%%%%%%


%%% Various Emacs customizations:
%%% Local Variables:
%%% mode: latex
%%% mode: LaTeX-math
%%% mode: reftex
%%% mode: Tex-PDF
%%% fill-column: 70
%%% indent-tabs-mode: t
%%% TeX-electric-sub-and-superscript: nil
%%% TeX-brace-indent-level: 0
%%% LaTeX-indent-level: 0
%%% LaTeX-item-indent: 0
%%% TeX-master: t
%%% End: