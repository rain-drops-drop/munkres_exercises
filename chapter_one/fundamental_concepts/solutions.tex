\documentclass[11pt]{article}

\oddsidemargin=17pt \evensidemargin=17pt
\headheight=9pt     \topmargin=26pt
\textheight=564pt   \textwidth=433.8pt

\usepackage{amsmath, amsthm, amssymb,graphicx,color,enumitem}%,mathrsfs}%,hyperref
\setlist[enumerate]{parsep=0pt plus 4pt,topsep=3pt plus 4pt}
\setlist[itemize]{parsep=0pt plus 4pt,topsep=3pt plus 4pt,itemsep=.5ex}

\leftmargini=5.5ex
\leftmarginii=3.5ex

%for \marginpar to fit optimally
%hoffset=-1.02in
\setlength\marginparwidth{2.2in}
\setlength\marginparsep{1mm}
\newcommand\red[1]{\marginpar{\linespread{.85}\sf%
	\vspace{-1.4ex}\footnotesize{\color{red}#1}}}
	%newcommand\score[1]{\marginpar{\colorbox{yellow}{#1/3}}\hspace{-1ex}}
	\newcommand\score[1]{\marginpar{\vspace{-2ex}\color{blue}{#1/3}}\hspace{-1ex}}
	\newcommand\extra[1]{\marginpar{\color{blue}{#1/1}}\hspace{-1ex}}
	\newcommand\total[2]{\marginpar{\colorbox{yellow}{\huge #1/#2}}}
	\newcommand\collab[1]{\marginpar{\vspace{-11ex}\colorbox{yellow}{#1/3}}\hspace{-1ex}}
	\newcommand\magenta[1]{\colorbox{magenta}{\!\!#1\!\!}}
	\newcommand\yellow[1]{\colorbox{yellow}{\!\!#1\!\!}}
	\newcommand\green[1]{\colorbox{green}{\!\!#1\!\!}}
	\newcommand\cyan[1]{\colorbox{cyan}{\!\!#1\!\!}}
	\newcommand\rmagenta[2][\vspace{0ex}]{\red{#1\magenta{\phantom{:}}\,: #2}}
	\newcommand\ryellow[2][\vspace{0ex}]{\red{#1\yellow{\phantom{:}}\,: #2}}
	\newcommand\rgreen[2][\vspace{0ex}]{\red{#1\green{\phantom{:}}\,: #2}}
	\newcommand\rcyan[2][\vspace{0ex}]{\red{#1\cyan{\phantom{:}}\,: #2}}
	
	%For separated lists with consecutive numbering
	\newcounter{separated}
	
	\newcommand{\excise}[1]{}
	\newcommand{\comment}[1]{{$\star$\sf\textbf{#1}$\star$}}
	\newcommand{\exnumber}[1]{\noindent\textbf{#1}}
	
	%%%%%%%%%%%%%%%%%%%%%%%%%%%%%%%%%%%%%%%%%%%%%%%%%%%
	%                                                 %
	% TO ENABLE GRADING, DO NOT ALTER ABOVE THIS LINE %
	%                                                 %
	%%%%%%%%%%%%%%%%%%%%%%%%%%%%%%%%%%%%%%%%%%%%%%%%%%%
	
	%new math symbols taking no arguments
	\newcommand\0{\mathbf{0}}
	\newcommand\CC{\mathbb{C}}
	\newcommand\FF{\mathbb{F}}
	\newcommand\NN{\mathbb{N}}
	\newcommand\QQ{\mathbb{Q}}
	\newcommand\RR{\mathbb{R}}
	\newcommand\ZZ{\mathbb{Z}}
	\newcommand\bb{\mathbf{b}}
	\newcommand\kk{\Bbbk}
	\newcommand\mm{\mathfrak{m}}
	\newcommand\xx{\mathbf{x}}
	\newcommand\yy{\mathbf{y}}
	\newcommand\GL{\mathit{GL}}
	\newcommand\pset{\mathcal{P}}
	\newcommand\into{\hookrightarrow}
	\newcommand\nsub{\trianglelefteq}
	\newcommand\onto{\twoheadrightarrow}
	\newcommand\minus{\smallsetminus}
	\newcommand\goesto{\rightsquigarrow}
	
	%redefined math symbols taking no arguments
	\newcommand\<{\langle}
	\renewcommand\>{\rangle}
	\renewcommand\iff{\Leftrightarrow}
	\renewcommand\phi{\varphi}
	\renewcommand\implies{\Rightarrow}
	
	%new math symbols taking arguments
	\newcommand\ol[1]{{\overline{#1}}}
	
	%redefined math symbols taking arguments
	\renewcommand\mod[1]{\ (\mathrm{mod}\ #1)}
	
	%roman font math operators
	\DeclareMathOperator\im{im}
	
	%for easy 2 x 2 matrices
	\newcommand\twobytwo[1]{\left[\begin{array}{@{}cc@{}}#1\end{array}\right]}
	
	%for easy column vectors of size 2
	\newcommand\tworow[1]{\left[\begin{array}{@{}c@{}}#1\end{array}\right]}
	
	
	%%%%%%%%%%%%%%%%%%%%%%%%%%%%%%%%%%%%%%%%%%%%%%%%%%%%%%%%%%%%%%%%%%%%%%
	\begin{document}%%%%%%%%%%%%%%%%%%%%%%%%%%%%%%%%%%%%%%%%%%%%%%%%%%%%%%
%%%%%%%%%%%%%%%%%%%%%%%%%%%%%%%%%%%%%%%%%%%%%%%%%%%%%%%%%%%%%%%%%%%%%%


\title{\mbox{}\\[-8ex]Fundamental Concepts Solutions\normalsize
	\\[-2.5ex]}
\author{Solutions by: Raindrops\_drop \\[1ex]
	 \\[-1ex]}
\date{February 12, 2026} 
\maketitle

\vspace{-3ex}%
\noindent

\exnumber{Exercises}

\exnumber{1.} Check the distributive laws for \(\cup\) and \(\cap\) and DeMorgan's laws.

\begin{proof}
	We first show the distributive laws for the union and intersection set operations.
	Let \(A, B, C\) be sets. Thus, the laws we aim to prove are:
	\begin{enumerate}
		\item[(a)] \(A \cup (B \cap C) = (A \cup B) \cap (A \cup C)\).
		\item[(b)] \(A \cap (B \cup C) = (A \cap B) \cup (A \cap C)\). 
	\end{enumerate}
	
	We begin by showing \((a)\). If \(A\) were empty, then \(A \cup (B \cap C) = B \cap C\).
	Furthermore, \((A \cup B) \cap (A \cup C) = B \cap C\). Thus, our property holds. Similarly,
	if either \(B\) or \(C\) were empty, then \(B \cap C = \emptyset\). Thus, \(A \cup (B \cap C) = A\)
	and \((A \cup B) \cap (A \cup C) = A \cap A = A\). Thus, our property holds if either \(B\) or \(C\)
	are empty. In particular, we conclude that this equality holds if either or all of the sets are 
	empty. Thus we will now assume that \(A, B\), and \(C\) are all non-empty.
	
	Let \(x \in A \cup (B \cap C)\). Then, either \(x \in A\), \(x \in B \cap C\) or \(x\) is in 
	both sets are true. If \(x \in A\), then notably \(x \in A \cup B\) and \(x \in A \cup C\),
	and therefore \(x \in (A \cup B) \cap (A \cup C)\). Now, consider \(x \in B \cap C\). Then,
	by definition \(x \in B\) and \(x \in C\). In particular, \(x \in A \cup B\) and \(x \in A \cup C\),
	and therefore, \(x \in (A \cup B) \cap (A \cup C)\). Thus, we can conclude that 
	\(A \cup (B \cap C) \subseteq (A \cup B) \cap (A \cup C)\).
	
	Let \(y \in (A \cup B) \cap (A \cup C)\). Then, by definition \(y \in A \cup B\) and 
	\(y \in A \cup C\). Then, either \(y \in A, y \in B, y \in C\), or \(y\) is an element of all the
	sets or any combination of two of the sets. If \(y \in A\), note then that \(y \in A \cup (B \cap C)\).
	Thus, assume that \(y \notin A\). Then it must be that \(y \in B\) and \(y \in C\). This is true 
	because of the fact that \(y \in A \cup B\) and \(y \in A \cup C\). Thus, \(y \in B \cap C\),
	and therefore, \(y \in A \cup (B \cap C)\). Therefore, we can conclude that 
	\((A \cup B) \cap (A \cup C) \subseteq A \cup (B \cap C)\). As such, we've shown that 
	\(A \cup (B \cap C) = (A \cup B) \cap (A \cup C)\).
	
	We shall now prove \((b)\). If $A$ we empty, then $A \cap (B \cup C) = \emptyset$. Furthermore,
	$(A \cap B) \cup (A \cap C) = \emptyset$. Thus, the identity holds. Assume now that $B$ is empty.
	Then, $B \cup C = C$. Thus, $A \cap (B \cup C) = A \cap C$, and $(A \cap B) \cup (A \cap C) = A \cap C$.
	So, the identity holds, and similarly for the case where $C$ is empty. Note then that this identity 
	will hold if any of the sets is empty, and in particular it will also hold if any pair of sets or all
	the sets are empty. Thus, we will assume that $A, B,$ and $C$ are non-empty.
	
	Let $x \in A \cap (B \cup C)$. Then, $x \in A$ and $x \in B \cup C$. Since $x in B \cup C$, 
	either $x \in B, x \in C$, or $x$ is in both sets. If $x \in B$, then $x \in A \cap B$ and thus
	$x \in (A \cap B) \cup (A \cap C)$. Similarly, if $x \in C$, then $x \in A \cap C$ and thus 
	$x \in (A \cap B) \cup (A \cap C)$. Therefore, we can conclude that $A \cap (B \cup C) \subseteq 
	(A \cap B) \cup (A \cap C)$.
	
	Let $y \in (A \cap B) \cup (A \cap C)$. Then, either $y \in A \cap B$, $y \in A \cap C$, or 
	$y$ is an element of both sets. Assume that $y \in A \cap B$, then $y \in A$ and $y \in B$.
	In particular, since $y \in B$, it must also be true that $y \in B \cup C$. Thus,
	$y \in A \cap (B \cup C)$. Similarly, if $y \in A \cap C$, then $y \in A$ and $y \in C$.
	Since $y \in C$, it is also true that $y \in B \cap C$. Thus, $y \in A \cap (B \cup C)$.
	Thus, we conclude that $(A \cap B) \cup (A \cap C) \subseteq A \cap (B \cup C)$.
	Therefore, we have shown that \(A \cap (B \cup C) = (A \cap B) \cup (A \cap C)\). 
	
	We will now focus on the DeMorgan's laws. Which are as follows:
	\begin{enumerate}
		\item[(c)] $A - (B \cup C) = (A - B) \cap (A - C)$.
		\item[(d)] $A - (B \cap C) = (A - B) \cup (A - C)$.
	\end{enumerate}
	
	We will first prove $(c)$. Note that the identity holds if either $A = \emptyset$, $B = \emptyset$,
	or $C = \emptyset$. In particular, it will also hold if all three sets or any combination of two
	of the sets are empty. Thus, we will assume that all three sets are non-empty.
	
	Let $x \in A - (B \cup C)$. Then, by definition, $x \in A$ and $x \notin B \cup C$. Since 
	$x \notin B \cup C$, we note that it must be true that $x \notin B$ and $x \notin C$. 
	Thus, we note that by definition, $x \in A - B$ and $x \in A - C$. Therefore, $x \in (A - B) \cap 
	(A - C)$. As such, we can conclude that $A - (B - C) \subseteq (A - B) \cap (A - C)$.
	
	Let $y \in (A - B) \cap (A - C)$. Then, $x \in A - B$ and $x \in A - C$. In particular, we note 
	that $x \in A$. Furthermore, $x \notin B$ and $x \notin C$, which implies that $x \notin B \cup C$.
	Thus, $x \in A - (B \cup C)$; therefore, we have shown that $(A - B) \cap (A - C) \subseteq A - 
	(B \cup C)$. Thus, we conclude that $A - (B \cup C) = (A - B) \cap (A - C)$.
	
	We will now focus on proving $(d)$. Note that if $A = \emptyset$ our identity will still hold.
	If $B = \emptyset$, then $A - (B \cap C) = A$. Further note that $(A - B) \cup (A - C) = 
	A \cup (A - C)$. Since $A - C \subseteq A$, we note that $A \cup (A - C) = A$. Thus, our identity
	holds if $B = \emptyset$. If $C = \emptyset$, using a similar argument as with $B$, we also note
	that the identity holds. In particular, if any combination of two of the three sets or all three 
	sets are empty the identity will hold. Thus, we will assume that all of the sets are non-empty.
	
	Let $x \in A - (B \cap C)$. Then, $x \in A$ and $x \notin B \cap C$. Since $x \notin B \cap C$,
	either $x \notin B, x \notin C$ or $x$ is not an element of both sets. Assume that $x \notin B$,
	then $x \in A - B$; furthermore, $x \in (A - B) \cup (A - C)$. Similarly, if $x \notin C$,
	then $x \in A - C$, and thus $x \in (A - B) \cup (A - C)$. Thus, we have shown that 
	$A - (B \cap C) \subseteq (A - B) \cup (A - C)$.
	
	Let $y \in (A - B) \cup (A - C)$. Then, either $y \in A - B, y \in A - C$, or $y$ is an element 
	of both sets. Note that $y \in A$ in all of the possibilities. Thus, $y \notin B, y \notin C$,
	or $y$ is not an element of either set. In particular, note that in either of the possibilities,
	it must be that $y \notin B \cap C$. Thus, $y \in A - (B \cap C)$. Therefore, we have shown
	that $(A - B) \cup (A - C) \subseteq A - (B \cap C)$. Thus, we have shown that 
	$A - (B \cap C) = (A - B) \cup (A - C)$.
\end{proof}

\newpage

\exnumber{2.} Determine which of the following statements are true for all sets $A,B, C$ and $D$. 
If a double implication fails, determine whether one or the other of the possible implications holds.
If an equality fails determine whether the statement becomes true if the "equals" symbol is replaced
by one or the other of the inclusion symbols $\subset$ or $\supset$.

\begin{enumerate}[label=(\alph*)]
	\item $A \subset B$ and $A \subset C \iff A \subset (B \cup C)$.
	\begin{proof}
		Assume that $A \subset B$ and $A \subset C$. Note that if either $B$ or $C$ are empty,
		then it must be that $A = \emptyset$. As such, $A \subset (B \cup C)$. Thus, we assume that 
		neither $B$ or $C$ are empty. If $A = \emptyset$, then trivially $A \subset (B \cup C)$.
		Thus, we further assume that $A \neq \emptyset$. Let $a \in A$. Then, by definition, 
		$a \in B$ and $a \in C$. Thus, $a \in (B \cup C)$. Therefore, if $A \subset B$ and 
		$A \subset C$, then $A \subset (B \cup C)$.
		
		Assume that $A \subset (B \cup C)$. Note that if $B = \emptyset$ and both $A$ and $C$ are
		nonempty, then $B \cup C = C$, and $A \subset C$. Since $A$ is non-empty, it cannot be
		that $A \subset B$ since $B$ is empty. Thus, it is not true, in general, that 
		if $A \subset (B \cup C)$ then $A \subset B$ and $A \subset C$.
		
		Therefore, we can only conclude that $A \subset B$ and $A \subset C \implies A \subset (B \cup C)$.
	\end{proof}
	\item $A \subset B$ or $A \subset C \iff A \subset (B \cup C)$. 
	\begin{proof}
		Assume that $A \subset B$ or $A \subset C$. If $A = \emptyset$, then $A \subset  (B \cup C)$.
		Assume $B = \emptyset$, if $A \subset B$, then $A = \emptyset$ and thus $A \subset (B \cup C)$.
		If $A \subset C$, then $A \subset (B \cup C)$. We arrive at the same conclusion if $C = \emptyset$
		In particular, if any pair of these three sets or all three of them are empty, then we can
		conclude that $A \subset (B \cup C)$. Therefore, we will assume that all three of these sets 
		are non-empty.
		
		Assume that $A \subset B$. Then, if $a \in A$, it must be true that $a \in B$. Thus,
		$a \in B \cup C$, and we can conclude that $A \subset (B \cup C)$. Similarly, if $A \subset C$,
		and $a \in A$, we have that $a \in C$. Thus, $a \in B \cup C$, and thus $A \subset (B \cup C)$.
		
		Now, assume that $A \subset (B \cup C)$. For this counterexample, assume that $A, B$, and $C$
		are all non-empty. Furthermore, assume that $B - C$ and $C - B$ are nonempty. Since $A \subset 
		(B \cup C)$, let $a_1, a_2 \in A$ such that $a_1 \in B - C$ and $a_2 \in C - B$. Note then that
		$A \not\subset B$ and $A \not\subset C$ since $a_1 \not\in C$ and $a_2 \not\in B$. Thus,
		we cannot conclude in general that if $A \subset (B \cup C)$, then $A \subset B$ or $A \subset C$.
		Therefore, we can only conclude that $A \subset B$ or $A \subset C \implies A \subset (B \cup C)$.
	\end{proof}
	\item $A \subset B$ and $A \subset C \iff A \subset (B \cap C)$.
	\begin{proof}
		Assume that $A \subset B$ and $A \subset C$. Note that if either $B$ or $C$ are the empty set,
		then $A$ must be the empty set. In particular, it would imply that $A \subset (B \cap C)$.
		Similarly if $A$ were the empty set. Furthermore, if any two of the three sets are the empty
		set, or all three of the sets are empty, then $A \subset (B \cap C)$. Therefore, we will
		assume that all three sets are non-empty.
		
		Let $a \in A$. Then, by our assumption, $a \in B$ and $a \in C$. Thus, by definition,
		$a \in B \cap C$. So, we conclude that $A \subset (B \cap C)$.
		
		Now, assume that $A \subset (B \cap C)$. Note that if either of the sets is empty,
		then $A \subset B$ and $A \subset B$. Furthermore, if any two of the three sets or all sets
		are empty, then $A \subset B$ and $A \subset C$. Thus, we will assume that all three sets are 
		non-empty.
		
		Let $a \in A$. Then, by our assumption, $a \in (B \cap C)$. Thus, $a \in B$ and $a \in C$.
		Thus, by definition of a subset, $A \subset B$ and $A \subset C$. Therefore, we conclude that 
		$A \subset B$ and $A \subset C \iff A \subset (B \cap C)$.
	\end{proof}
	\item $A \subset B$ or $A \subset C \iff A \subset (B \cap C)$.
	\begin{proof}
		Assume that $A \subset B$ or $A \subset C$. Note that if all sets are non-empty, $A \subset B$,
		$B \cap C = \emptyset$, then $A \not\subset (B \cap C)$. Thus, it is not true in general that 
		if $A \subset B$ or $A \subset C$, then $A \subset (B \cap C)$.
		
		Assume that $A \subset (B \cap C)$. Then, by $(c)$, $A \subset B$ and $A \subset C$.
		In particular, $A \subset B$ or $A \subset C$ is also true. Therefore, we've shown that 
		$A \subset B$ or $A \subset C \impliedby A \subset (B \cap C)$. 
	\end{proof} 
	\item $A - (A - B) = B$.
	\begin{proof}
		This equality is not true in general. To show this we will first show three results. The first
		of such results is the fact that $A - B = A \cap B^{c}$. 
		
		First note that this result is true if either $A$ or $B$ or both are the empty set. Thus,
		we will assume that neither are empty.
		Let $x \in A - B$. Then, $x \in A$ and $x \notin B$. By definition, $x \in B^{c}$. Thus,
		it must be that $x \in A \cap B^{c}$. Therefore, $A - B \subseteq A \cap B^{c}$.
		
		Let $y \in A \cap B^{c}$. Then, $x \in A$ and $x \in B^{c}$. Since $x \in B^{c}$, it must be 
		that $x \notin B$. Thus,by definition $x \in A - B$. Therefore, $A \cap B^{c} \subseteq A -B$.
		As such, we have shown that $A - B = A \cap B^{c}$.
		
		The next result we will show is the fact that $(A^{c})^{c} = A$. If $A = \emptyset$, then
		$A^{c} = X$ where $X$ is the ambient set under which $A$ is defined. In particular,
		the compliment of the set is the empty set. Thus, $(A^{c})^{c} = \emptyset = A$. Thus our 
		result holds. Therefore, we will assume that $A$ is non-empty.
		
		Let $a \in (A^{c})^{c}$. By definition, $a \notin A^{c}$. Since $a \notin A^{c}$, it must 
		be that $a \in A$. Thus, $(A^{c})^{c} \subseteq A$. Now, let $a \in A$. Then, by definition,
		$a \notin A^{c}$. Since $a \notin A^{c}$, it must be that $a \in (A^{c})^{c}$. Therefore,
		$A \subseteq (A^{c})^{c}$. As such, $A = (A^{c})^{c}$.
		
		The final result we need is the following: $(A \cap B)^c = A^c \cup B^c$. 
		Note that if $A = \emptyset$, $()A \cap B)^c = (\emptyset)^c = X$, where $X$
		is the ambient set. Furthermore, $A^c \cup B^c = X \cup B^c = X$. Thus, 
		the result is true. We arrive at the same conclusion through an analogous argument
		if $B = \emptyset$ or both sets are empty. Thus, we will assume that both sets 
		are non-empty.
		
		Let $x \in (A \cap B)^c$. By definition, $x \notin A \cap B$. Thus, either $x \notin
		A$, $x \notin B$, or $x$ is not in both sets. If $x \notin A$, then $x \in A^c$, as 
		such $x \in A^c \cup B^c$. Therefore, $(A \cap B)^c \subseteq A^c \cup B^c$. We 
		arrive at the same conclusion in the other two cases through an analogous 
		argument.
		
		Let $x \in A^c \cup B^c$. Then, either $x \in A^c$, $x \in B^c$, or $x$ is in both
		sets. If $x \in A^c$, then $x \notin A$. In particular, this implies that 
		$x \notin A \cap B$. Thus, $x \in (A \cap B)^c$ and therefore $A^c \cup B^c
		\subseteq (A \cap B)^c$. We arrive at the same conclusion through analogous arguments
		for the other cases. Hence, we conclude that $(A \cap B)^c = A^c \cup B^c$.  
		
		With these three results we can now prove our original claim.
		\begin{align*}
			A - (A - B) &= A \cap (A - B)^{c} \\
						&= A \cap (A^{c} \cup (B^{c})^{c}) \\ 
						&= A \cap (A^{c} \cup B) \\
						&= (A \cap A^{c}) \cup (A \cap B) \\
						&= A \cap B		
		\end{align*}
		Hence, $A - (A - B) = A \cap B$. In particular, it implies that $A - (A - B) \subset B$.
	\end{proof}
	\item $A - (B - A) = A - B$.
	\begin{proof}
		We will make use of the results proved in the previous problem.
		\begin{align*}
			A - (B - A) &= A \cap (B - A)^c \\
						&= A \cap (B \cap A^c)^c \\
						&= A \cap (B^c \cup (A^c)^c) \\
						&= A \cap (B^c \cup A) \\
						&= (A \cap B^c) \cup (A \cap A) \\
						&= (A - B) \cup A \\
						&= A
		\end{align*}
		Note that $A - B \subseteq A$. As such, it must be that the claim in this problem
		is not true in general and thus, $A - (B - A) \supseteq A - B$
	\end{proof}
	\item  $A \cap (B - C) = (A \cap B) - (A \cap C)$.
	\begin{proof}
		If $A$ is empty, note that $A \cap (B - C) = \emptyset$ and $(A \cap B) - (A \cap C) = 
		\emptyset$. Thus the claim is true. If $B$ is empty, then $(B - C) = \emptyset$,
		and thus $A \cap (B - C) = \emptyset$. Furthermore, $A \cap B = \emptyset$, and as 
		such $(A \cap B) - (A \cap C) = \emptyset$. Thus, the claim is true. Similarly,
		if $C = \emptyset$, then $A \cap (B - C) = A \cap C$. $(A \cap B) - (A \cap C) = 
		A \cap B$. Thus, the claim is true. In particular, if any two of the three
		sets are empty, or all three sets are empty the claim will be true. As such, we 
		will assume that all three sets are non-empty. 
		
		Let $x \in A \cap (B - C)$. Then, $x \in A$ and $x \in B - C$. Since $x \in B - C$,
		$x \in B$ and $x \notin C$. In particular, we note that $x \in A \cap B$.
		Since $x \notin C$, we have that $x \notin A \cap C$. Therefore, 
		$x \in (A \cap B) - (A \cap C)$. Hence, we conclude that 
		$A \cap (B - C) \subseteq (A \cap B) - (A \cap C)$.
		
		Let $x \in (A \cap B) - (A \cap C)$. Then, $x \in A \cap B$ and $x \notin A \cap C$.
		Since $x \in A \cap B$, $x \in A$ and $x \in B$. Furthermore, since $x \notin A \cap C$
		and $x \in A$, it must be that $x \notin C$. In particular, we have that 
		$x \in B - C$ by definition and $x \in A$. Therefore, $x \in A \cap (B - C)$.
		Hence, $(A \cap B) - (A \cap C) \subseteq A \cap (B - C)$. Thus, we can
		conclude that $A \cap (B - C) = (A \cap B) - (A \cap C)$.
	\end{proof}
	\item $A \cup (B - C) = (A \cup B) - (A \cup C)$.
	\begin{proof}
		Note that if $B = \emptyset$, then $A \cup (B - C) = A$. Furthermore,
		$(A \cup B) - (A \cup C) = A - (A \cup C) = \emptyset$. Thus, the equality does 
		not hold in general since $\emptyset \subset A$ and not equal. Hence, we will
		aim to prove instead that $A \cup (B - C) \supseteq (A \cup B) - (A \cup C)$.
		If $A$ is empty, note that $A \cup (B - C) = B - C$. Furthermore, 
		$(A \cup B) - (A \cup C) = B - C$. Thus, our claim holds. If $C$ is empty,
		then $A \cup (B - C) = A \cup B$. Furthermore, $(A \cup B) - (A \cup C) = 
		(A \cup B) - A = B - A$. Note $B - A \subseteq A \cup B$. Thus, our claim holds.
		In particular, if any two of the three sets are empty or all three sets are empty,
		the claim will hold. Thus, we will assume that all three sets are non-empty.
		
		Let $x \in (A \cup B) - (A \cup C)$. Then, $x \in A \cup B$ and $x \notin A \cup C$.
		Since $x \in A \cup B$, either $x \in A$, $x \in B$, or $x$ is an element of both
		sets. Furthermore, since $x \notin A \cup C$, then $x \notin A$ and $x \notin C$.
		Since $x \notin A$, it must be that $x \in B$. By definition, we have that 
		$x \in B - C$. In particular, we also have that $x \in A \cup (B - C)$.
		Therefore, we conclude that $A \cup (B - C) \supseteq (A \cup B) - (A \cup C)$.
	\end{proof}
	\item $(A \cap B) \cup (A - B) = A$.
	\begin{proof}
		Let $A = \emptyset$, then $A \cap B = \emptyset$, and $A - B = \emptyset$,
		as such, $(A \cap B) \cup (A - B) = \emptyset$. Thus, the claim holds. 
		If $B = \emptyset$, then $A \cap B = \emptyset$, and $A - B = A$. Thus,
		$(A \cap B) \cup (A - B) = A$. Thus, our claim holds. Also note that the claim
		holds if both sets are empty. Thus, we will assume that both sets are non-empty.
		
		Let $a \in (A \cap B) \cup (A - B)$. By definition, either $a \in A \cap B$,
		$a \in A - B$, or $a$ is an element of both sets. If $a \in A \cap B$, we note 
		that $a \in A$. If $a \in A - B$ we also have that $a \in A$. Therefore, 
		since in all cases $a \in A$, we conclude that $(A \cap B) \cup (A - B) \subseteq  A$.
		
		Let $a \in A$. If $A \cap B = \emptyset$, then note that $a \in A - B$ and therefore,
		$a \in (A \cap B) \cup (A - B)$. Thus, assume that $A \cap B \neq \emptyset$. Note that 
		if $a \in A$ and $a \notin A \cap B$, we also have that $a \in A - B$. Since 
		$A \cap B \neq \emptyset$ there exists some $a \in A$ such that $a \in A \cap B$ also.
		Therefore, $a \in (A \cap B) \cup (A - B)$. Note that there cannot exist some $a \in A$
		such that $a \notin A - B$ and $a \notin A \cap B$. This is because if $a \in A$, but
	and $D$ are non-empty and $B \cap D$ is non-empty. 	$a \notin A - B$, it must be that $a \in B$ and therefore $a \in A \cap B$.
		Similarly, if $a \in A$ and $a \notin A \cap B$, then $a \notin B$ and therefore
		$a \in A - B$. Thus, we have shown that $A \subseteq (A \cap B) \cup (A - B)$.
		Therefore, we can conclude that $(A \cap B) \cup (A - B)$.
	\end{proof}
	\newpage
	\item $A \subset C$ and $B \subset D \implies (A \times B) \subset (C \times D)$.  
	\begin{proof}
		Let $A \subset C$ and $B \subset D$. Consider the set $A \times B$. If $A$ is empty,
		note that $A \times B = \emptyset$ and thus $A \times B \subset C \times D$.
		We arrive at the same conclusion if $B$ is empty or both sets are empty through
		an analogous argument. Thus, we will assume that both $A$ and $B$ are non-empty.
		
		Let $x \in A \times B$. By definition, $x = (a, b)$ such that $a \in A$ and $b \in B$.
		By assumption, we have that $a \in C$ and $b \in D$ since $A \subset C$ and 
		$B \subset D$. Therefore, we have that $(a, b) \in C \times D$. Thus, we conclude
		that $A \times B \subset C \times D$.
	\end{proof}
	\item The converse of $(i)$. If $A \times B \subset C \times D$, then 
	$A \subset C$ and $B \subset D$.
	\begin{proof}
		This statement is not true in general. If $A = \emptyset$, $B = \{1, 2\}$,
		$C = \{1\}$, and $D = \{2\}$, note that $A \times B = \emptyset$ is a subset
			of $C \times D$. However, $B$ is not a subset of $D$. 
	\end{proof}
	\item The converse of $(i)$, assuming that $A$ and $B$ are nonempty.
	\begin{proof}
		Let $A \times B \subset C \times D$ such that $A$ and $B$ are nonempty.
		As such, there exists some $x \in A \times B$ which, by definition, 
		has the property that $x = (a, b)$ where $a \in A$ and $b \in B$. Since 
		$A \times B \subset C \times D$, $(a, b) \in C \times D$. Therefore,
		$a \in C$ and $b \in D$ by definition. Thus, we can conclude that 
		$A \subset C$ and $B \subset D$.
	\end{proof}
	\item $(A \times B) \cup (C \times D) = (A \cup C) \times (B \cup D)$.
	\begin{proof}
		Note that if $A$ is empty, then $(A \times B) \cup (C \times D) = C \times D$.
		Furthermore, $(A \cup C) \times (B \cup D) = C \times (B \cup D)$. Note that 
		$C \subseteq C$ and $D \subseteq B \cup D$. Thus, by $(j)$ we have that 
		$C \times D \subseteq C \times (B \cup D)$. In particular if any of the sets is 
		empty we arrive at an analogous conclusion using a similar argument. If both 
		$A$ and $B$ are empty, note that both sides of the identity will be equal, 
		which will also happen if both $C$ and $D$ are empty. If $A$ and $C$ are empty,
		then both sides of the equality are empty and thus are equal, which will also happen
		if both $B$ and $D$ are empty. If $A$ and $D$ are empty, the left side will be empty,
		and the right side will be $C \times B$, note $\emptyset \subseteq C \times B$.
		We arrive at a similar conclusion if both $B$ and $C$ are empty through an 
		analogous argument. If any three of the four sets are empty, then both sides
		of the identity will be empty, similarly if all four sets are empty.
		
		From this, we've noted that this identity will fail if $A$ is empty, $B,C,$ and $D$
		are non-empty such that $B - D \neq \emptyset$. Thus, $C \times D \subset 
		C \times (B \cup D)$. Since $B - D$ is nonempty, there exists some $d \in D$
		such that $d \notin B$, Thus, this is a strict inclusion and therefore the equality
		does not hold. As such, it is not true in general that $(A \cup C) \times (B \cup D)
		\subseteq (A \times B) \cup (C \times D)$. We may check for the opposite direction
		of the inclusion.
		
		Assume that all sets are non-empty, as we have already checked the cases where 
		each set is empty and they all agree that $(A \times B) \cup (C \times D) \subseteq 
		(A \cup C) \times (B \cup D)$.
		
		Let $x \in (A \times B) \cup (C \times D)$. By definition, either $x \in A \times B$,
		$x \in C \times D$, or $x$ is an element of both sets. Let $x = (n, m)$. If 
		$x \in A \times B$, then $n \in A$ and $m \in B$. In particular, $n \in A \cup C$
		and $m \in B \cup D$. Therefore, $x \in (A \cup C) \times (B \cup D)$. We arrive 
		at the same conclusion if $x \in C \times D$ through an analogous argument.
		Therefore, we can conclude that $(A \times B) \cup (C \times D) \subseteq 
		(A \cup C) \times (B \cup D)$.
	\end{proof}
	\item $(A \times B) \cup (C \times D) = (A \cap C) \times (B \cap D)$.
	\begin{proof}
		Firstly, note that if any of the sets are empty then both sides of the claim 
		will be empty and the equality will hold. In particular, this will be true 
		for any combination of the sets being empty. As such, we shall examine this 
		claim for when all the sets are non-empty.
		
		Let $x \in (A \times B) \cup (C \times D)$ such that $x = (n, m)$. Then, 
		$x \in A \times B$ and $x \in C \times D$. Since $x \in A \times B$, $n \in A$
		and $m \in B$. Furthermore, since $x \in C \times D$, $n \in C$ and $m \in D$.
		Since $n \in A$ and $n \in C$, $n \in A \cap C$. Similarly, since $m \in B$
		and $m \in D$, $m \in B \cap D$. By definition, it must be that 
		$x \in (A \cap C) \times (B \cap D)$. Therefore, we conclude that 
		$(A \times B) \cap (C \times D) \subseteq (A \cap C) \times (B \cap D)$.
		
		Let $y \in (A \cap C) \times (B \cap D)$ such that $y = (n, m)$. By definition,
		$n \in A \cap C$ and $m \in B \cap D$. Since $n \in A \cap C$, it must be 
		that $n \in A$ and $n \in C$. Similarly, since $m \in B \cap D$, it must also be
		that $m \in B$ and $m \in D$. In particular, $x \in A \times B$ and 
		$x \in C \times D$ by definition. Therefore, $x \in (A \times B) \cap (C \times D)$.
		Hence, it must be that $(A \cap C) \times (B \cap D) \subseteq (A \times B)
		\cap (C \times D)$. Therefore, we conclude that $(A \times B) \cap (C \times D) = 
		(A \cap C) \cap (B \times D)$.
	\end{proof}
	\item $A \times (B - C) = (A \times B) - (A \times C)$.
	\begin{proof}
		Note that if either $A$ or $B$ are empty, then both sides of this claim will
		be empty allowing the equality to hold. If $C$ is empty note that both sides
		of the claim will be equivalent to $A \times B$. If any two of the three
		sets or all three of the sets note that both sides of the claim will be empty.
		Thus, the equality will hold for all of these cases. As such, we will focus now
		on the case where all of the sets are non-empty.
		
		Let $x \in A \times (B - C)$ such that $x = (n, m)$. By definition,
		$n \in A$ and $m \in B - C$. Note that $m \in B$ and $m \notin C$. 
		In particular, since $m \notin C$ it must be that $x \notin A \times C$.
		Since $n \in A$ and $m \in B$ we have that $x \in A \times B$. Thus, we 
		conclude that $x \in (A \times B) - (A \times C)$. Therefore, 
		$A \times (B - C) \subseteq (A \times B) - (A \times C)$.
		
		Let $y \in (A \times B) - (A \times C)$ such that $y = (n. m)$. 
		By definition, $y \in A \times B$ and $y \notin A \times C$.
		Since $y \in A \times B$, $n \in A$ and $m \in B$. Furthermore,
		since $y \notin A \times C$, either $n \notin A$, $m \notin C$, or both are 
		not in the corresponding set. Since $n \in A$, it must be that $m \notin C$.
		Thus, not that $m \in B$ and $m \notin C$, which implies that $m \in B - C$.
		Therefore, we have that $y \in A \times (B - C)$. As such, we conclude that
		$(A \times B) - (A \times C) \subseteq A \times (B - C)$. Hence, we have shown 
		that $A \times (B - C) = (A \times B) - (A \times C)$.
	\end{proof}
	\item $(A - B) \times (C - D) = (A \times C - B \times C) - A \times D$.
	\begin{proof}
		Firstly, note that if either $A$ or $C$ are empty, both sides of this claim are 
		empty and as such are equal. If $B$ is empty, the claim will reduce to 
		$A \times (C - D) = (A \times C) - (A \times D)$ which is true by $(o)$. 
		If $D$ is empty, the claim will reduce to $(A - B) \times C = (A \times C)
		- (B \times C)$, which is true by $(o)$. If $B$ and $D$ are empty,
		note that the claim will reduce to $A \times C = A \times C$ which is true.
		If any other combination of two of the four sets are empty, then both sides
		of the claim will be empty. Similarly, if any combination of three of the four
		sets or all of the four sets are empty, then both sides of the claim will be true.
		Thus, the only case left to check is when all of the four sets are non-empty.
		
		Let $x \in (A - B) \times (C - D)$ such that $x = (n, m)$. By definition,
		$n \in A - B$ and $m \in C - D$. Since $n \in A - B$, we have that 
		$n \in A$ and $n \notin B$. Similarly, it must be that $m \in C$ and $m \notin D$.
		In particular, note that $x \in A \times C$ and that $x \notin B \times C$
		since $n \notin B$. Therefore, $x \in (A \times C) - (B \times C)$. Furthermore,
		since $m \notin D$, $x \notin A \times D$. Thus, we conclude that 
		$x \in (A \times C - B \times C) - A \times D$.
		
		Let $y \in (A \times C - B \times C) - A \times D$ such that $y = (n, m)$. 
		By definition, $n \in A$ and $m \in C$. Since $y \notin B \times C$ and 
		$m \in C$ it must be that $n \notin B$. Furthermore, since $y \notin A \times D$
		and $n \in A$, it must be that $m \notin D$. In particular, since $n \in A$ and 
		$n \notin B$, $n \in A - B$. Further, since $m \in C$ and $m \notin D$, 
		$m \in C - D$. Therefore, $y \in (A - B) \times (C - D)$. Hence, we conclude that
		$(A \times C - B \times C) - A \times D \subseteq (A - B) \times (C - D)$.
		Therefore, we conclude that $(A - B) \times (C - D) = (A \times C - B \times C)
		- A \times D$. 
	\end{proof}
	\item $(A \times B) - (C \times D) = (A - C) \times (B - D)$.
	\begin{proof}
		Note that if either $A$ or $B$ are empty, then both sides of the claim are empty
		and the equality holds. If $C$ is empty, then the claim will become
		$A \times B = A \times (B - D)$, which in general note that $A \times B
		\supseteq A \times (B - D)$. In particularly, the equality does not hold
		if both $B$, $D$, and $B \cap D$ are non-empty. If $D$ is empty, the claim
		reduces to $A \times B = (A - C) \times B$. Using a similar argument as 
		with the previous case, we arrive that only $\supseteq$ can be true in general
		for this case. If both $C$ and $D$ are empty, note that the claim reduces 
		to $A \times B = A \times B$ which is true. If any other combination of two
		of the four sets are empty, both sides of the claim are empty and thus the claim is 
		true. Similarly, any combination of three of the four sets or all four of the 
		sets are empty, then both sides of the equality are empty. Thus, since in general
		only $\supseteq$ may be true for this claim, we need only check that direction
		for the case where all four sets are non-empty.
		
		Let $x \in (A - C) \times (B - D)$ such that $x = (n, m)$. By definition,
		$n \in A$ and $n \notin C$, and $m \in B$ and $m \notin D$. Since 
		$n \in A$ and $m \in B$, $x \in A \times B$. Furthermore, since 
		$n \notin C$ and $m \notin D$, $x \notin C \times D$. As such, 
		$x \in (A \times B) - (C \times D)$. Therefore, $(A \times B) - (C \times D) 
		\supseteq (A - C) \times (B - D)$.
	\end{proof} 
\end{enumerate}

\exnumber{3.} 
\begin{enumerate}[label=(\alph*)]
	\item Write the contrapositive and converse of the following statement: "If $x < 0$,
	then $x^2 - x > 0$," and determine which (if any) of the three statements are true.
	\begin{proof}
		Firstly, we will show the main statement is true. Note that if $x < 0$, then
		$-x > 0$. Furthermore, by properties of the field of real numbers, $x^2 > 0$.
		Therefore, $x^2 - x > 0$.
		
		The contrapositive of the statement will be: "If $x^2 - x \leq 0$, then $x \geq 0$."
		Since the contrapositive of a statement is logically equivalent to the original
		statement, and in this case, the original statement is true, it must be that
		this statement is also true.
		
		The converse of the statement is: "If $x^2 - x > 0$, then $x < 0$." Note that this 
		statement is not true in general. Let $x^2 - x = 2 > 0$. Then, note that 
		either $x = 2$ or $x = -1$. Since it is possible $x = 2 > 0$, the statement 
		is not true.
	\end{proof}
	\item Do the same for the statement "If $x > 0$, then $x^2 - x > 0$.
	\begin{proof}
		First, we will show that the main statement is not true in general. 
		Consider the case $x = 1/2$. Then, $x^2 - x = -1/4 < 0$.
		
		The contrapositive of the statement is: "If $x^2 - x \leq 0$, then $x \leq 0$."
		Since the contrapositive of a statement is logically equivalent to the original
		statement we can conclude that this claim is not true since the original claim
		was not true.
		
		The converse of the statement is: "If $x^2 - x > 0$, then $x > 0$." This claim
		is not true in general. To show this simply consider the same example as in
		$(a)$ in which $x^2 - x = 2 > 0$.
	\end{proof}
\end{enumerate}

\exnumber{4.} Let $A$ and $B$ be sets of real numbers. Write the negation of each of the 
following statements.
\begin{enumerate}[label=(\alph*)]
	\item For every $a \in A$, it is true that $a^2 \in B$.
	
	The negation of this statement is: For at least one $a \in A$, it is true 
	that $a^2 \notin B$.
	\item For at least one $a \in A$, it is true that $a^2 \in B$.
	
	The negation of this statement is: For every $a \in A$, it is true that $a^2 \notin B$.
	\item For every $a \in A$, it is true that $a^2 \notin B$. 
	
	The negation of this statement is: For at least one $a \in A$, it is true that
	$a^2 \in B$.
	\item For at least one $a \notin A$, it is true that $a^2 \in B$.
	
	The negation of this statement is: For all $a \notin A$, it is true that 
	$a^2 \notin B$.
\end{enumerate}

\pagebreak
\exnumber{5.} Let $\mathcal{A}$ be a nonempty collection of sets. Determine the truth
of each of the following statements and of their converses:
\begin{enumerate}[label=(\alph*)]
	\item $x \in \bigcup_{A \in \mathcal{A}} A \implies x \in A$ for at least one 
	$A \in \mathcal{A}$.
	
	\begin{proof}
		Let $x \in \bigcup_{A \in \mathcal{A}} A$. Then, by definition 
		$x \in A$ for at least one $A \in \mathcal{A}$. 
		
		Let $x \in A$ for at least one $A \in \mathcal{A}$. By definition, $x \in 
		\bigcup_{A \in \mathcal{A}} A$. 
		
		Therefore, by definition this claim and its converse are true.
	\end{proof}
	
	\item $x \in \bigcup_{A \in \mathcal{A}} A \implies x \in A$ for every $A \in \mathcal{A}$.
	
	\begin{proof}
		Let $x \in \bigcup_{A \in \mathcal{A}} A$. By definition, $x \in A$ for at least
		one $A \in \mathcal{A}$. Let $A$ be the $A \in \mathcal{A}$ such that 
		$x \in A$. In particular, assume there exists some $A' \in \mathcal{A}$
		such that $x \notin A'$. Since $x \in A$ it is still true that 
		$x \in \bigcup_{A \in \mathcal{A}} A$ even though $x \notin A'$. Thus,
		this statement is not true in general.
		
		Let $x \in A$ for every $A \in \mathcal{A}$. In particular, this implies that
		$x \in A$ for at least one $A \in \mathcal{A}$. Therefore, by definition,
		$x \in \bigcup_{A \in \mathcal{A}}A$. As such the converse of the claim is true.
	\end{proof}
	
	\item $x \in \bigcap_{A \in \mathcal{A}} A \implies x \in A$ for at least one 
	$A \in \mathcal{A}$.
	
	\begin{proof}
		Let $x \in \bigcap_{A \in \mathcal{A}} A$. By definition, $x \in A$ for every
		$A \in \mathcal{A}$. In particular, since $x \in A$ for every $A \in \mathcal{A}$,
		it is also true that $x \in A$ for at least one $A \in \mathcal{A}$.
		
		Let $x \in A$ for at least one $A \in \mathcal{A}$. In particular, there may
		exist some $A' \in \mathcal{A}$ such that $x \notin A'$. As such,
		it cannot be that in general $x \in \bigcap_{A \in \mathcal{A}} A$.
		Therefore, the converse of the statement is not true in general.
	\end{proof}
	
	\item $x \in \bigcap_{A \in \mathcal{A}} A \implies x \in A$ for every $A \in \mathcal{A}$.
	
	\begin{proof}
		Let $x \in \bigcap_{A \in \mathcal{A}} A$. By definition $x \in A$ for every
		$A \in \mathcal{A}$.
		
		Let $x \in A$ for every $A \in \mathcal{A}$. By definition 
		$x \in \bigcap_{A \in \mathcal{A}} A$. Thus, the converse of the statement is 
		also true.
	\end{proof}
\end{enumerate}

\exnumber{6.} Write the contrapositive of each of the statements of Exercise 5.
\begin{enumerate}[label=(\alph*)]
	\item $x \notin A$ for every $A \in \mathcal{A} \implies 
	x \notin \bigcup_{A \in \mathcal{A}} A$.
	\item $x \notin A$ for at least one $A \in \mathcal{A} \implies x \notin 
	\bigcup_{A \in \mathcal{A}} A$.
	\item $x \notin A$ for every $A \in \mathcal{A} \implies 
	x \notin \bigcap_{A \in \mathcal{A}} A$.
	\item $x \notin A$ for at least one $A \in \mathcal{A} \implies 
	x \notin \bigcap_{A \in \mathcal{A}} A$.
\end{enumerate}

\pagebreak
\exnumber{7.} Given sets $A, B$, and $C$, express each of the following sets in terms of 
$A, B$, and $C$, using the symbols $\cup, \cap, -$.
\begin{enumerate}[label=(\alph*)]
	\item $D = \{x \mid x \in A \text{ and } (x \in B) \text{ or } x \in C\}$.
	
	$D = A \cap (B \cup C)$.
	\item $E = \{x \mid (x \in A \text{ and } x \in B) \text{ or } x \in C\}$.
	
	$E = (A \cap B) \cup C$.
	\item $F = \{x \mid x \in A \text{ and } (x \in B \implies x \in C)\}$.
	
	$F = A - (B - C)$.
\end{enumerate}

\exnumber{8.} If a set $A$ has two elements, show that $\pset(A)$ has four elements.
How many elements does $\pset(A)$ have if $A$ has one element? Three elements? 
No elements? Why is $\pset(A)$ called the power set of $A$?

Let $A$ be a set of two elements, $A = \{a, b\}$. Note that the power set of $A$
is equal to $\{\{\}, \{a\}, \{b\}, \{a, b\}\}$. which has four elements. If 
$A$ has one element, say $A = \{a\}$, then the power set of $A$ will be the set 
$\{\{\}, \{a\}\}$, which has two elements. If $A$ has three elements, say 
$A = \{a, b, c\}$ the power set of $A$ will be 
$\{\{\}, \{a\}, \{b\}, \{c\}, \{a, b\}, \{a, c\}, \{b, c\}, \{a, b, c\}\}$, which 
has eight elements. If $A$ has no elements, note that its power set will be the set 
$\{\{\}\}$ which has one element.

We note that in each of these cases, the cardinality of $\pset(A)$ has been
$2^n$ where $n$ is the cardinality of $A$. As such, that is why $\pset(A)$
is called the power set of $A$.

\exnumber{9.} Formulate and prove DeMorgan's laws for arbitrary unions and intersections.

Let $A$ be a set and $\mathcal{B}$ be a collection of sets.
\begin{enumerate}[label=(\alph*)]
	\item $A - \bigcup_{B \in \mathcal{B}} B = \bigcap_{B \in \mathcal{B}} (A - B)$.
	
	\begin{proof}
		Note that if $A$ is empty, both sides of the identity are empty, and as such are 
		equal. If $\mathcal{B}$ is a collection of empty sets, note that both
		sides of the identity will be equivalent to $A$, and as such the equality holds.
		If $A$ is empty and $\mathcal{B}$ is a collection of empty sets, then the identity
		will hold. Thus, we will assume that $A$ is non-empty, and that $\mathcal{B}$ contains
		at least one set $B$ that is not empty.
		
		If $A - \bigcup_{B \in \mathcal{B}} B$ is empty, then it is trivially a subset 
		of $\bigcap_{B \in \mathcal{B}} (A - B)$. Thus, we will assume that
		$A - \bigcup_{B \in \mathcal{B}} B$ is non-empty. Let 
		$x \in A - \bigcup_{B \in \mathcal{B}} B$. Then $x \in A$ and 
		$x \notin \bigcup_{B \in \mathcal{B}} B$. In particular, since 
		$x \notin \bigcup_{B \in \mathcal{B}} B$, by definition $x \notin B$
		for any $B \in \mathcal{B}$. Thus, $x \in A - B$ for all $B \in \mathcal{B}$.
		Therefore, $x \in \bigcap_{B \in \mathcal{B}} (A - B)$, and we can conclude that
		$A - \bigcup_{B \in \mathcal{B}} B \subseteq \bigcap_{B \in \mathcal{B}} (A - B)$.
		
		If $\bigcap_{B \in \mathcal{B}} (A - B)$ is empty,  then it is trivially a subset
		of $A - \bigcup_{B \in \mathcal{B}} B$. Thus we will assume that it is non-empty.
		Let $x \in \bigcap_{B \in \mathcal{B}} (A - B)$. By definition, $x \in A - B$
		for all $B \in \mathcal{B}$. Thus, $x \in A$ and $x \notin B$ for all 
		$B \in \mathcal{B}$ Therefore, $x \in A - \bigcup_{B \in \mathcal{B}} B$
		by definition. Hence, $\bigcap_{B \in \mathcal{B}} (A - B) \subset 
		A - \bigcup_{B \in \mathcal{B}} B$.
		
		Therefore, we conclude that $A - \bigcup_{B \in \mathcal{B}} B = 
		\bigcap_{B \in \mathcal{B}} (A - B)$.
	\end{proof}
	\pagebreak
	\item $A - \bigcap_{B \in \mathcal{B}} B = \bigcup_{B \in \mathcal{B}} (A - B)$.
	
	\begin{proof}
		Note that if $A$ is empty, both sides of the identity are empty, and as such
		are equal. If at least one $B \in \mathcal{B}$ is empty, then 
		$\bigcap_{B \in \mathcal{B}} B = \emptyset$. Thus, in this case we have 
		that the identity becomes $A = A$, which holds true. In particular, if 
		$A$ is empty and at least one $B \in \mathcal{B}$ is empty, then the 
		identity will hold. Thus, we will assume that $A$ is non-empty, and that
		every $B \in \mathcal{B}$ is non-empty.
		
		If $A - \bigcap_{B \in \mathcal{B}} B$ is empty, then it is trivially a subset
		of $\bigcup_{B \in \mathcal{B}} (A - B)$. Thus, we will assume that 
		$A - \bigcap_{B \in \mathcal{B}} B$ is non-empty. Let 
		$x \in A - \bigcap_{B \in \mathcal{B}} B$. By definition, $x \in A$
		and $x \notin \bigcap_{B \in \mathcal{B}} B$. Since 
		$x \notin \bigcap_{B \in \mathcal{B}} B$, then $x \notin B$ for at least one 
		$B \in \mathcal{B}$. In particular, $x \in A - B$ for such $B \in \mathcal{B}$.
		Note that $A - B \in \bigcup_{B \in \mathcal{B}} (A - B)$. Therefore,
		$x \in \bigcup_{B \in \mathcal{B}} (A - B)$ and we can conclude that 
		$A - \bigcap_{B \in \mathcal{B}} B \subseteq \bigcup_{B \in \mathcal{B}} (A - B)$.
		
		If $\bigcup_{B \in \mathcal{B}} (A - B)$ is empty, then it is trivially a subset
		of $A - \bigcap_{B \in \mathcal{B}} B$. Thus, we will assume that 
		$\bigcup_{B \in \mathcal{B}} (A - B)$ is non-empty. Let 
		$x \in \bigcup_{B \in \mathcal{B}} (A - B)$. By definition, $x \in A - B$
		for at least one $B \in \mathcal{B}$. In particular, $x \in A$ and 
		$x \notin B$ for at least one $B$. Since $x \notin B$ for at least one 
		$B \in \mathcal{B}$, it must be that $x \notin \bigcap_{B \in \mathcal{B}} B$.
		Therefore, since $x \in A$ and $x \notin \bigcap_{B \in \mathcal{B}} B$,
		we have that $x \in A - \bigcap_{B \in \mathcal{B}} B$. Hence, we conclude
		that $\bigcup_{B \in \mathcal{B}} (A - B) \subseteq A - \bigcap_{B \in \mathcal{B}} B$.
		As such, we conclude that $A - \bigcap_{B \in \mathcal{B}} B = 
		\bigcup_{B \in \mathcal{B}} (A - B)$.
	\end{proof}
\end{enumerate}

\exnumber{10.} Let $\RR$ denote the set of real numbers. For each of the following 
subsets of $\RR \times \RR$, determine wether it is equal to the cartesian product
of two subsets of $\RR$.
\begin{enumerate}[label=(\alph*)]
	\item $\{(x, y) \mid x \text{ is an integer}\}$. 
	
	Note that if $(x, y)$ is in this set, then it must be that $x \in \RR$ and 
	$y \in \ZZ$. Both of these sets are subsets of $\RR$.
	\item $\{(x, y) \mid 0 \leq y \leq 1\}$.
	
	Note that if $(x, y)$ are in the set, then $x \in \RR$ and $y \in [0, 1]$. Both of 
	these sets are subsets of $\RR$.
	\item $\{(x, y) \mid y > x\}$.
	
	Note that this set is not made up of two distinct subsets of $\RR$. This is because
	both $x, y \in \RR$ but the set is not equivalent to $\RR^2$.
	\item $\{(x, y) \mid x \text{ is not an integer and } y \text{ is an integer}\}$.
	
	Note that if $(x, y)$ is in the set, then $x \in \RR - \ZZ$ and $y \in \ZZ$. Both of 
	which are subsets of $\RR$.
	\item $\{(x, y) \mid x^2 + y^2 < 1\}$.
	
	Note that this set is not made up of two distinct subset of $\RR$. This is because
	both $x, y \in (0, 1)$ but the set is not equivalent to $(0, 1) \times (0, 1)$.
\end{enumerate}
%%%%%%%%%%%%%%%%%%%%%%%%%%%%%%%%%%%%%%%%%%%%%%%%%%%%%%%%%%%%%%%%%%%%%%
\end{document}%%%%%%%%%%%%%%%%%%%%%%%%%%%%%%%%%%%%%%%%%%%%%%%%%%%%%%%%
%%%%%%%%%%%%%%%%%%%%%%%%%%%%%%%%%%%%%%%%%%%%%%%%%%%%%%%%%%%%%%%%%%%%%%


%%% Various Emacs customizations:
%%% Local Variables:
%%% mode: latex
%%% mode: LaTeX-math
%%% mode: reftex
%%% mode: Tex-PDF
%%% fill-column: 70
%%% indent-tabs-mode: t
%%% TeX-electric-sub-and-superscript: nil
%%% TeX-brace-indent-level: 0
%%% LaTeX-indent-level: 0
%%% LaTeX-item-indent: 0
%%% TeX-master: t
%%% End: